  \Wrap{
    content/english={
      The \emph{SageMath} application gives the whole structure of integer
      numbers as the \texttt{ZZ} object which behaves according to the "usual"
      mathematical definition.},
    content/magyar ={
      A \emph{SageMath} programcsomagban az egész számokat a \texttt{ZZ}
      objektummal kapjuk meg a ,,szokásos``  matematikai definíciónak
      megfelel\H en.}
  }
  \begin{sageexample}
    sage: type(ZZ)
  \end{sageexample}
  
  \Wrap{
    content/magyar={
      Természetesen nem kell minden esetben használnunk a konstruktort, ha egy
      egész számmal szeretnénk dolgozni, a rendszer automatikusan felismeri.
    },
    content/english={
      Fortunatelly there is no need to use the constructor when we want to
      create an integer, the system automaticly recognizes such numbers. 
    }
  }
  \begin{sageexample}
    sage: a,b = ZZ(4), 4
    sage: type(a) == type(b)
    sage: a == b
  \end{sageexample}

  \Wrap{
    content/magyar={Aritmetiaki m\H uveletek a ,,szokásosak``:},
    content/english={Arithmetic on integers:}
  }
  \begin{itemize}
    \item 
      \Wrap{
        content/magyar={Összeadás, kivonás},
        content/english={Addition, substraction}
      }: \texttt{+,-};
    \item 
      \Wrap{
        content/magyar={Szorzás, hatványozás},
        content/english={Multiplication, raising to a power}
      }: \texttt{*,\^};
    \item 
      \Wrap{
        content/magyar={Egész érték\H u osztás és maradékképzés},
        content/english={Division over the integers and residue}
      }: \texttt{//,\%}.
  \end{itemize}
  
  \Wrap{
    content/magyar={
      \emph{Megjegyzés:} A \texttt{/} m\H uvelet eredménye egy racionális
      szám, sőt valójában a jelenléte elég, hogy innentől racionálisként
      tekintsen a megadott adatokra.},
    content/english={
      \emph{Note:} The result of the \texttt{/} operand is a rational number
      and in general if this operand  is used then the result's type will 
      be rational.}
  }
  \begin{sageexample}
    sage: 2/3
    sage: type(2/3)
    sage: 1/1
    sage: type(1/1)
  \end{sageexample}

\begin{Section}{
  title/english=Divisor,
  title/magyar=Oszthatóság}

  \begin{definition}[\Wrap{content/magyar=Osztó, content/english=Divisor}]
  \Wrap{
    content/magyar={Az $a$ osztója $b$-nek és $b$ többszöröse $a$-nak, azaz $a|b$, ha},
    content/english={$a$ is a divisor of $b$ and $b$ is a multiple of $a$ that is to say
      $a|b$, if}
  }
  \[ \exists c: b = ac.\]
\end{definition}

\begin{exercise}
  \Wrap{
    content/magyar={
      Írd meg azt a függvény, amely edönti, hogy az első argumentuma osztható-e a másodikkal az alábbi
      példán kívűl még 4 különböző módon!
    },
    content/english={
      Write a function that decides whether its first argument can be divided with its second argument 4
      different way beside the example below!
    }
  }
  \begin{sageexample}
    sage: def divides0(a,b):
    ....:     return (a/b).is_integer()
    sage: divides0(5,2)
    sage: divides0(6,3)
  \end{sageexample}

  \begin{solution}
    \Wrap{
      content/magyar={Sok megoldás elképzelhető, például:},
      content/english={There are a lot of possible solutions, a few for example:}
    }
    \begin{sageexample}
        sage: def divides0(a,b):
        ....:     return (a/b).is_integer()
        sage: def divides1(a,b):
        ....:     return a % b == 0
        sage: def divides2(a,b):
        ....:     return (a//b)*b == a
        sage: def divides3(a,b):
        ....:     return (a/b).denom() == 1
        sage: def divides4(a,b): #there is room to improve
        ....:     if a == 0:
        ....:         return True
        ....:     b *= sign(b)
        ....:     if b == 1:
        ....:         return True
        ....:     q = b
        ....:     a *= sign(a)
        ....:     while q <= a:
        ....:         q <<= 1
        ....:     while a > b:
        ....:         q >>= 1
        ....:         a -= q
        ....:         a *= sign(a)
        ....:     return a == 0 or a == b

      \end{sageexample}
  \end{solution}
\end{exercise}

\emph{
  \Wrap{
    content/magyar={Oszthatóság tulajdonságai természetes számok esetén:},
    content/english={Properties of divison over natural numbers:}
  }
}
\begin{enumerate}
  \item \Wrap{ content/magyar={Részbenrendezés, azaz}, content/english={Partial ordering, i.e.}}
    \begin{itemize}
      \item \Wrap{content/magyar=Reflexív, content/english=Reflexive}
        ($\forall a\in\mathbb{Z}: a|a$),
      \item \Wrap{ content/magyar=Antiszimmetrikus, content/english=Antisymmetric}
        ($\forall a,b\in\mathbb{N}: a|b\wedge b|a\Rightarrow a=b$),
      \item \Wrap{content/magyar=Tranzitív, content/english=Transitive}
        ($\forall a,b,c\in\mathbb{Z}: a|b \wedge b|c \Rightarrow a|c$);
    \end{itemize}
  \item \Wrap{
    content/magyar=minden szám osztja $0$-t,
    content/english=every number divides $0$};
  \item $1$ \Wrap{
    content/magyar=minden számnak osztója,
    content/english=divides every number};
  \item $0$ \Wrap{
    content/magyar=csak saját magának osztója,
    content/english=only divides $0$};
  \item $a|b \wedge c|d \Rightarrow ac|bd$;
  \item $a|b \Rightarrow \forall k\in \mathbb{Z}: ak|bk$;
  \item $k\in \mathbb{N}\setminus{0}:ak|bk \Rightarrow a|b$;
  \item $a|b \wedge a|c \Rightarrow a|b+c$;
  \item \Wrap{
    content/magyar=egy pozitív szám minden osztója kisebb vagy egyenlő mint a szám maga.,
    content/english=every divisor of a number is less or equal then the number itself.}
\end{enumerate}

\Wrap{
  content/magyar={A részbenrendezések, így az oszthatóság is megadható egy speciális objektummal},
  content/english={In \emph{SageMath} a partial ordering can be represented with a special object}}:
\texttt{Poset} (\emph{P}artially \emph{O}rdered \emph{S}et).
\Wrap{
  content/magyar={Az ilyen objektumot ábrázolva kapjuk a részberendezések szemléltetésére használt Hasse-diagramot.}
  content/english={By the graphical representation of such an object, we get a Hasse-diagram.}}
\begin{sageexample}
  sage: k = 15
  ....: P = Poset((Set([2..k]), lambda a,b: b % a == 0))
\end{sageexample}
\begin{figure}[h!]
  \centering
  \sageplot[scale=.6][png]{P.plot(talk=True)}
  \Caption{
    text/magyar={Hasse-diagram oszthatóság esetén $2$ és $k$ közötti természetes számokon. (\texttt{P.plot(talk=True)})},
    text/english={Hasse-diagram for division for natural numbers between $2$ and $k$. (\texttt{P.plot(talk=True)})}
  }
\end{figure}

\begin{exercise}
  \Wrap{
    content/magyar={Írj programot, amely egy adott számhalmaz esetén megszámolja hány él van az oszthatóság
      relációhoz tartozó  Hasse-diagramban! Ellenőrzésre lehet használni az alábbi kódot.},
    content/english={Write program that counts how many edges are in the division relation's Hasse-diagram of
      of a given set. The below command can be used for testing the output.}
  }

  \begin{sageexample}
    sage: len(P.cover_relations_graph().edges())
  \end{sageexample}

  \begin{solution}
    \Wrap{
      content/magyar={Faktorizáció és gráfok nélkül.},
      content/english={Without factorization and gaphs.}
    }
    \begin{sageexample}
      sage: def edges_in_div_hasse(S):
      ....:     count = 0
      ....:     L = list(S)
      ....:     L.sort(reverse=True)
      ....:     for i in range(1,len(L)):
      ....:         S = set()
      ....:         for j in range(i-1,-1,-1):
      ....:             if L[j] % L[i] == 0:
      ....:                 direct = True
      ....:                 for k in S:
      ....:                     if L[j] % L[k] == 0:
      ....:                         direct = False
      ....:                         break
      ....:                 if direct:
      ....:                     S.add(j)
      ....:                     count += 1
      ....:     return count
    \end{sageexample}
  \end{solution}
\end{exercise}

\begin{exercise}
  \Wrap{
    content/magyar={Írj programot, amely egy adott egész szám esetén kiírja osztóinak számát, illetve
      osztóinak összegét! Ellenőrzéshez használhatjuk a \texttt{sigma(n,0)} és \texttt{sigma(n,1)} parancsokat.},
    content/english={Write a program that write out the number of divisors and the sum of divisors for a given
      integer. You can use \texttt{sigma(n,0)} and \texttt{sigma(n,1)} for testing.}
  }

  \begin{solution}
    \begin{sageexample}

    \end{sageexample}
  \end{solution}
\end{exercise}

\begin{exercise} Írj programot, amely a természetes számok egy adott
  halmazában megkeresi a tökéletes számokat (tökéletes szám: osztóinak
  összege megegyezik a számmal, pl. 6).
\end{exercise}

\begin{exercise}{Aliquot} Természetes számok esetén definiálhatjuk a következő
  sorozatot: $(s_0 = n; s_{i+1} = \sigma(s_i) - s_i)$, ahol a $\sigma(n)$
  az $n$ osztóinak összege. A sorozat vagy terminál nulla értékkel vagy
  periódikussá válik. Készíts programot, amely egy adott természetes szám
  esetén kiszámolja az említett sorozatot. (Ha nem terminál, akkor csak az
  első periódust írja ki.)
\end{exercise}

\begin{definition}[Asszociált] Az $a\neq b$ elemek asszociáltak, ha $a|b$ és
  $b|a$ is teljesül.
\end{definition}

\begin{definition}[Egység] Egy $e$ elem egységelem, ha bármely $a$ elemre
  $a=ea=ae$. Az egységelem asszociáltjait egységeknek hívjuk.
\end{definition}

\begin{definition}[Irreducibilis] Egy nem egység $a$ elemet felbonthatatlannak vagy
  irreducibilisnek nevezünk, ha $a=bc$ esetén $b$ és $c$ közül az
  egyik egység.
\end{definition}

\begin{definition}[Prím] Egy nem egység $p$ elemet prímnek nevezünk, ha $p|ab$
  esetén a $p|a$ vagy a $p|b$ közül legalább az egyik teljesül.
\end{definition}

Megjegyzések:
\begin{enumerate}
  \item Az egység és egységelem két külön fogalom, egységelem egyedi,
    amíg az is előfordulhat, hogy a struktúra összes eleme egység (pl.: $\mathbb{Q}$).
  \item Az egység alternatív definíciója: $a$ egység, ha bármely $b$ elem
    felírható $b=ac$ alakban.
  \item Minden prímelem egyben irreducibilis is, hiszen \[
      p = ab \Rightarrow p|ab \Rightarrow p|b \Rightarrow b = qp \Rightarrow
      p = (aq)p \Rightarrow a \text{ egység}.
    \]
  \item Természetes számok esetén (és minden Gauss-gy\H ur\H uben), ha egy
    elem irreducibilis, akkor prím is.
  \item Lehet olyan struktúrát mutatni, ahol van olyan irreducibilis elem,
    ami nem prímelem. Például ha tekintjük az egész konstans taggal rendelkező
    egyváltozós polinomokat, azaz a $\mathbb{Z} + x\mathbb{R}[x]$ struktúrát,
    akkor az $x$ felbonthatatlansága nyilvánvaló; ugyanakkor az
    $x|(x\sqrt{2})^2$ teljesül, de $x$ nem osztja $x\sqrt{2}$-t, ui. az
    osztás eredményének is benne kellene lennie a struktúrában, de
    $\sqrt{2}\notin\mathbb{Z}$.
\end{enumerate}

\begin{exercise} A $\mathbb{Z}_m$ struktúra alatt a ${0,1,\dots, m-1}$
  számokat értjük úgy, hogy az összeadás és szorzás műveletet $\mod m$ értjük.
  Írj programot amely egy adott $m$ esetén definíció alapján meghatározza az
  egységeket, irreducibiliseket és prímeket!
\end{exercise}

Természetes számok esetén nyilvánvaló, hogy végtelen sok prím van, hiszen ha
feltennénk, hogy véges sok van, akkor azokat összeszorozva és az eredményt
eggyel megnövelve olyan számot kapnánk, aminek egyik sem osztója.
A prímek számára becslést az $\frac{x}{\ln x}$ formulával kaphatunk, amíg
\emph{SageMath}-ban a \texttt{prime\_pi(x)} függvénnyel kaphatjuk meg a
pontos számukat.
\begin{sageexample}
  sage: P1 = plot(x/log(x), (2, 200), scale='semilogy', \
  ....:     fill=lambda x: prime_pi(x),fillcolor='red')
  ....: P2 = plot(1.13*log(x), (2, 200), \
  ....:     fill=lambda x: nth_prime(x)/floor(x), fillcolor='red')
  ....: P = graphics_array([P1, P2])
\end{sageexample}
\begin{figure}[ht]
  \centering
  \sageplot[scale=.6]{P,figsize=[8,4]}
  \caption{Prímek számának és növekedésének becslése (\texttt{P1,P2}) }
\end{figure}

\begin{theorem}[Számelmélet alaptétele] Minden pozitív természetes szám a
  sorrendtől eltekintve egyértelműen felírható prímszámok szorzataként.
\end{theorem}

\begin{exercise}[Erasztotenész szitája] Adj programot, amely megadja az
  összes prímet egy adott számig, azaz ugyanazt az eredmény adja mint a
  $\mathtt{primes\_first\_n(n)}$!
\end{exercise}

\begin{exercise} Írd meg az előző feladatot hatékonyabban úgy, hogy a páros
  számok ne is kerüljenek be a táblába!
\end{exercise}

\begin{exercise} Írd meg a prímszitát úgy, hogy a $2,3$ és $5$-tel osztható
  számok ne kerüljenek a táblába! Ehhez a számokat $30i+M[j]$ alakban tárold
  ($30=2\cdot 3\cdot 5$), ahol $i\in[1,\lceil n\rceil]$; $j\in[1,8]$ és
  $M=[1,7,11,13,17,19,23,29]$.
\end{exercise}

\begin{exercise}[Ikerprímek]
    Természetes számok esetén az olyan prímeket melyeknek különbsége 2
    ikerprímeknek hívjuk. Írj programot, amely megkeresi az összes ikerprímet
    adott $a$ és $b$ között.
\end{exercise}

\begin{definition}[Legnagyobb közös osztó]
  Az $a$ és $b$ legnagyobb közös osztója az a $c = (a,b)$, amelyre
  \[ c|a \wedge c|b \wedge \forall d: d|a \wedge d|b \Rightarrow d|c. \]
  Ha a struktúrában van ,,szokásos`` rendezés (ilyen az egész számok), akkor ezek
  közül csak a legnagyobbat tekintjük legnagyobb közös osztónak. (Például a 12
  és 18 egész számokra a 6 és -6 is megfelelő lenne, de (12,18)=6.)
\end{definition}

\begin{exercise}
  Írj programot, kiszámolja a legnagyobb közös osztót a $\mathtt{factor}$
  parancs segítségével! Tesztelésre használható a $\mathtt{gcd(a,b)}$ parancs.
\end{exercise}

\begin{definition}[Legkisebb közös többszörös]
  Az $a$ és $b$ legkisebb közös többszöröse az a $c$, amelyre $c\cdot(a,b)=ab$.
\end{definition}

\begin{exercise}
  Írj programot, kiszámolja a legkisebb közös többszöröst a $\mathtt{factor}$
  parancs segítségével (és gcd használata nélkül)! Tesztelésre használható a
  $\mathtt{lcm(a,b)}$ parancs.
\end{exercise}

\begin{definition}[Relatív prím]
  Ha $(a,b)=1$, akkor $a$ és $b$ relatív prímek.
\end{definition}

\begin{definition}[Euklideszi algoritmus] A legnagyobb közös
  osztója $a$-nak és $b$-nek kiszámolható a következ\H o algoritmussal
  (amennyiben van maradékos osztás a struktúrában):
  \begin{enumerate}
    \item Legyen $a = qb + r$, ahol $0 \le r < b$.
    \item Ha $r = 0$, akkor a legnagyobb közös osztó $a$.
    \item $b \leftarrow a$
    \item $a \leftarrow  r$
    \item Ugorjunk (1)-re.
  \end{enumerate}
\end{definition}

\begin{exercise}
  Készítsd el a fenti algoritmust és hasonlítsd össze a korábbi legnagyobb
  közös osztót számoló program futási idejével!
\end{exercise}

\begin{exercise}[Binary GCD] Írj programot a legnagyobb közös osztó kiszámolására, ami
  csak additív és shift műveleteket használ (hatékony
  számítógépen) az alábbi összefüggéseket használva!
  \begin{itemize}
    \item $(2a,2b) = 2(a,b)$,
    \item $(2,b)=1 \Rightarrow (2a,b) = (a,b)$,
    \item $(a,b) = (a-b,b)$ és így ha $a$ és $b$ is páratlan, akkor $a-b$
      páros.
  \end{itemize}
\end{exercise}

\begin{theorem}
  Létezik olyan $x$ és $y$, amelyekre \[ ax+by=(a,b). \]
\end{theorem}

\begin{definition}[B\H ovített Euklideszi algoritmus]
  Az $(a,b)$ és a hozzá tartozó $x$, $y$ értékek ($(a,b)=ax+by$) meghatározására szolgáló
  algoritmus. A hagyományos algoritmushoz hasonlóan a maradékokat ($r_i$)
  fogjuk számolni az \[r_i = r_{i-2} - q_ir_{i-1}\] alakban, továbbá használjuk
  az \[
    \begin{array}{rcl}
      ax_i+by_i & = & r_i \\
                & = & r_{i-2} - q_ir_{i-q} \\
                & = & (ax_{i-2} + bx_{i-2}) - q_i(ax_{i-1} + by_{i-1}) \\
                & = & a(x_{i-2} - q_ix_{i-1}) + b(y_{i-2} - q_iy_{i-1})
    \end{array}
  \] invariánst. Ennek eleget téve az algoritmus
  \begin{enumerate}
    \item $x_0, y_0, r_0 \leftarrow 1, 0, a$;
    \item $x_1, y_1, r_1 \leftarrow 0, 1, b$;
    \item $i \leftarrow 1$
    \item Ha $r_i = 0$ akkor a megoldás $(x_i,y_i,r_i)$, különben $i\leftarrow i+1$;
    \item $q_i \leftarrow \lfloor r_{i-2}/r_{i-1}\rfloor$
    \item $x_i, y_i, r_i \leftarrow x_{i-2}-q_ix_{i-1}, y_{i-2}-q_iy_{i-1},
      r_{i-2}-q_ir_{i-1}$
    \item Ugorjunk (4)-re.
  \end{enumerate}
\end{definition}

\begin{exercise} Írj programot, ami a bővített Euklideszi algoritmust
  valósítja meg természetes számokra! Ellen\H orzéshez használható az
  $\mathtt{xgcd}$ parancs.
\end{exercise}

\begin{definition}[(Lineáris) Diofantikus probléma]
  Az $a,b,c\in\mathbb{Z}$ számok esetén az $ax+by=c$ egyenletet az egész
  számok fölött (egész megoldásokat keresünk) lineáris Diofantikus
  egyenletnek hívunk.
\end{definition}

A megoldások számának vizsgálatánál először észrevehet\H o, hogy $(a,b)$
osztja a bal oldalt, hiszen $a$-nak és $b$=nek is osztója, így a jobb oldalt
is kell osztania. Ez azt jelenti, hogy csak akkor van megoldás, ha $(a,b)|c$.
Viszont ebben az esetben biztosan van megoldás hiszen a b\H ovített
Euklideszi algoritmussal kaphatunk  egyet, ha annak kimenetét megszorozzuk
$c/(a,b)$-vel ($x_0, y_0$). Ha van még további megoldás, akkor az felírható
az $(x_0 + x', y_0 + y')$ alakban alkalmas $x',y'$ számokkal. Ekkor \[
  a(x_0+x')+b(y_0+y') = c = ax_0 + by_0,\] azaz \[ ax' = -by'. \]
A jobb oldal osztható $b$-val, így a bal is, tehát \[
  \begin{array}{lclrl}
    \displaystyle b|ax' \Rightarrow \frac{b}{(a,b)}|x' & \Rightarrow &
    x'= & &  \displaystyle t\frac{b}{(a,b)} \\
    \displaystyle ax' = -by' \Rightarrow at\frac{b}{(a,b)} = -by' & \Rightarrow &
    y'= & - & \displaystyle t\frac{a}{(a,b)}\\
  \end{array}\qquad (t\in\mathbb{Z}).\]
Összefoglalva, ha van megoldás akkor végtelen sok van és egy tetsz\H oleges
$(x_0, y_0)$ megoldásból a többit a
\[
  x_t = x_0 + t\frac{b}{(a,b)}\qquad y_t= y_0-t\frac{a}{(a,b)}\qquad(t\in\mathbb{Z})\]
formulákkal kaphatjuk.

\emph{Megjegyzés:} A lineáris Diofantikus probléma elképzelhet\H o egy úgy
is, hogy az egyenlet egy egyenes egyenlete a síkon és a kérdés az, hogy
ennek az egyenesnek van-e és mennyi metszéspontja van az egész számok segítségével
készített ráccsal. A fenti megoldás itt annak felel meg, hogy megpróbáljuk
egész érték\H u eltolással elmozdítani az egyenes egy pontját az origóba. Ha
sikerül az az jelenti, hogy a ráccsal közös pontok( az origón kívül) a
meredekségnek $\frac{a}{b}=\frac{a/(a,b)}{b/(a,b)}$ megfelel\H o négyzetek
megfelel\H o csúcsai lesznek..

\begin{exercise} Valósítsd meg a $\mathtt{LinDiofantianEq}$ osztályt a
  következőeknek megfelelően!
  \begin{itemize}
    \item Konstruktorában kell megadni az $a,b,c$ értékeket.
    \item Van egy $\mathtt{is\_solvable}$ függvénye.
    \item Fel tudja sorolni a megoldásokat egy $\mathtt{next\_solution}$ és
      egy $\mathtt{prev\_solution}$ függvény segítségével.
    \item Az első megoldás, amivel a $\mathtt{next\_solution}$ visszatér az
      legyen, amely esetén az $x$ a legkisebb nemnegatív szám.
    \item Csak egy megoldást tároljunk az objektum használata közben.
  \end{itemize}
\end{exercise}

\begin{exercise}
  Hányféleképpen tudunk kifizetni 100000 peng\H ot 47 és 79 peng\H os
  érmékkel?
\end{exercise}


\begin{exercise}
  Egy üzletben háromféle csokoládé kapható, 70, 130 és 150 forint
  egységárban. Hányféleképpen lehet pontosan 5000 forintért 50
  darab csokoládét venni?
\end{exercise}

\begin{exercise}
  Írj programot, amely a három argumentuma ($a,b,c$) visszatér hány
  megoldása van az $ax+by=c$ diofantikus problémának a természetes
  számok felett (nemnegatív megoldások)!
\end{exercise}

\begin{exercise}
  Írd meg a \[ \mathtt{multi(L, c, s=0)} \]
  függvényt, amelyre
  \begin{itemize}
    \item $\mathtt{L}$ lista elemei $a_0,a_1,\dots$;
    \item visszatérési érték a \[ \sum_{i=0}^{\mathtt{len(L)}} a_ix_i = c \]
      egyenlet nemnegatív egész megoldásainak száma $\mathtt{s=0}$ esetén,
      különben
    \item azon megoldások száma, amelyek még teljesítik a
      \[ \sum_{i=0}^{\mathtt{len(L)}} x_i = s \]
      feltételt is.
  \end{itemize}
\end{exercise}


\end{Section}

\begin{Section}{
  title/english=Congruent,
  title/magyar=Kongruencia}

  \begin{definition}[\Wrap{content/magyar=Kongruencia,content/english=Congruent}]
  \Wrap{
    content/magyar={Az $a$ és $b$ számok kongruensek modulo $m$ ($m>0$), azaz
      \[ a\equiv b \mod m\text{, amennyiben }m|(a-b).\]},
    content/english={The numbers of $a$ and $b$ are congruent modulo $m$ ($m>0$), i.e.
      \[a\equiv b(\mod m)\text{, if }m|(a-b).\]}
  }
\end{definition}

\Wrap{
  content/magyar={A kongruencia mint reláció reflexív, szimmetrikus és tranzitív is, azaz
    ekvivalenciareláció, így meghatározza az alaphalmaz egy osztályozását.},
  content/english={The congruence as a relation is reflexive, symmetric and transitive, thus
    an equivalence relation, which implies a partitioning of the set its defined over.}
}

\begin{exercise}
  \Wrap{
    content/magyar={Írj programot, amely egy egész számokat tartalmazó halmaz elemeit osztályozza
      modulo $m$, ahol az $m$ a második paraméter.},
    content/english={Write a program that sorting the elements of a given set (first argument)
      modulo $m$ (second argument).}
  }

  \begin{solution}
    \begin{sageexample}
      sage: def residue_sets(S,m):
      ....:     rs = {}
      ....:     for e in S:
      ....:         r = e % m
      ....:         if r in rs.keys():
      ....:             rs[r].add(e)
      ....:         else:
      ....:             rs[r] = set([e])
      ....:     return rs
    \end{sageexample}
  \end{solution}
\end{exercise}

\begin{definition}[\Wrap{content/magyar={Maradékrendszer}, content/english={Residue system}}]
  \Wrap{
    content/magyar={Egész számok esetén a kongruencia mint ekvivalenciareláció által
      meghatározott osztályokat \emph{maradékosztály}nak, míg rendszerüket \emph{maradékrendszer}nek
      nevezzük.},
    content/english={TODO}
  }
\end{definition}

\Wrap{
  content/magyar={Számolás során a maradékosztályokat egy-egy reprezentánsukkal szoktuk jelölni,
    például $m$ esetén gyakori a $0,1,\dots,m-1$ (legkisebb nem negatív reprezentások) vagy
    egész számok esetén a $-\lfloor \frac{m-1}{2}\rfloor,\dots,0,\dots,\lceil\frac{m-1}{2}\rceil$
    (legkisebb abszolút értékű reprezentások) használata.},
  content/english={TODO}
}

\begin{definition}[\Wrap{content/magyar={Redukált maradékrendszer}, content/english={Reduced residue system}}]
  \Wrap{
    content/magyar={Ha a maradékrendszerből elhagyjuk az összes olyan maradékosztályt melyek elemei
      nem relatív prímek a modulushoz, akkor megkapjuk a \emph{redukált maradékrendszer}t.},
    content/english={TODO}
  }
\end{definition}

\begin{definition}[\Wrap{content/magyar={Euler-féle $\varphi$ függvény}}]
  \Wrap{
    content/magyar={A $\varphi:\mathbb{N}\to\mathbb{N}$ függvényt az Euler-féle $\varphi$
      függvénynek nevezzűk, ha $\varphi(m)$ a modulo $m$ redukált maradékrendszerek száma, azaz},
    content/english={TODO}
  }
  \[\varphi(m) = \left|\{k\in\mathbb{Z}:1\le k< m\wedge (k,m)=1\}\right|.\]
\end{definition}

\Wrap{
  content/magyar={Ha $p$ egy prím és $n$ tetszőleges természetes szám, akkor a $\varphi(p^n) =p^n-p^{n-1}$
    könnyen kapható, hiszen pontosan minden $p$-edik maradékosztály tartalmaz $p$-vel osztható
    számokat, a többiben relatív prímek vannak $p$-hez és így $p^n$-hez is. Össztett számokkal való
    számoláshoz elég észrevenni, hogy a $\varphi$ számelméleti függvény multiplikatív, azaz
    relatív prím $a,b$ számokra $\varphi(ab)=\varphi(a)\varphi(b)$.},
  content/english={TODO}
}

\Wrap{
  content/magyar={A $\varphi(n)$ maximuma nyilvánvalóan $n-1$, viszont minimuma közel sem lineáris.},
  content/english={TODO}
}


\begin{sageexample}
  sage: P  = points([(k,euler_phi(k)) for k in range(1,1001)])
\end{sageexample}
\begin{figure}[h]
  \centering
  \sageplot[scale=.6]{P}
  \caption{Euler-féle $\varphi$ függvény értéke 1 és 1000 közötti számokra (\texttt{P}).}
\end{figure}

\begin{exercise}
  \Wrap{
    content/magyar={Írj programotfüggvényt, amely az Euler-féle $\varphi$ függvény értékét
      számolja ki! Ellenőrzéshez használható az \texttt{euler\_phi} parancs.},
    content/english={TODO}
  }

  \begin{solution}
    \begin{sageexample}
    # according to the definition
    sage: def ephi_01(m):
    ....:     p = 0
    ....:     for i in range(1,m):
    ....:         p += gcd(i,m) == 1
    ....:     return p

    #according to the previous note
    sage: def ephi_02(m):
    ....:     p = 1
    ....:     for (a,b) in factor(m):
    ....:         p *= a^(b-1)*(a-1)
    ....:     return p

    \end{sageexample}
  \end{solution}
\end{exercise}

\begin{definition}[\Wrap{content/magyar=Lineáris kongruenciák, content/english=Linear congruence}]
  \Wrap{
    content/magyar={Az $a,b$ egész és $m$ pozitív egész számok esetén az \[ax\equiv b\ (m)\] alakú
      kifejezéseket \emph{lineáris kongruenciának} hívjuk.},
    content/english={TODO}.
  }
\end{definition}

\Wrap{
  content/magyar={A kongruencia és oszthatóság definíciókat használva kapjuk, hogy alkalmas $y$-al},
  content/english={TODO}
}
\[ ax\equiv b\ (m) \Leftrightarrow m|ax-b \Leftrightarrow ax-b = my \Leftrightarrow ax-my=b.\]
\Wrap{
  content/magyar={Ez azt jelenti, hogy egy lineáris kongruencia megoldását megkaphatjuk a
    megfelelő lineáris diofantikus probléma megoldásával. Továbbá},
  content/english={TODO}
}
\begin{itemize}
  \item $(a,m)|b$
    \Wrap{
      content/magyar=szükséges és elégséges feltétel a megoldás létezésére;,
      content/english={TODO}
    }
  \item
    \Wrap{
      content/magyar={$acx\equiv bc\ (cm)$ kongruencia megoldásait megkaphatjuk az
        $ax\equiv b\ (m)$ kongruenca megoldásával;},
      content/english={TODO}
    }
  \item
    \Wrap{
      content/magyar={$(a,m)=1$ esetén mindkét oldalt oszthatjuk $(a,b)$-vel;},
      content/english={TODO}
    }
  \item
    \Wrap{
      content/magyar={$(a,m)=1$ és $(b,m)=c$ esetén a $ax\equiv b\ (m)$ kongruencia megoldásait
        kaphatjuk a $ax\equiv b/c\ (m/c)$ kongruencia megoldásával.},
      content/english={TODO}
    }
\end{itemize}

\begin{exercise}
  \Wrap{
    content/magyar={Írj eljárást lineáris kongruenciák megoldására! Ellenőrzéshez használható a
      \texttt{solve\_mod} parancs.},
    content/english={TODO}
  }
  \begin{solution}
    \begin{sageexample}
    sage: def lin_cong(a,b,m):
    ....:     (d,x,y) = xgcd(a,m)
    ....:     return x % m/d
    \end{sageexample}
  \end{solution}
\end{exercise}

\begin{definition}[\Wrap{content/magyar={Moduláris inverz}, content/english=Modular inverse}]
  \Wrap{
    content/magyar={Az $ax\equiv 1\ (m)$ kongruencia megoldását (ha van) az $a$ szám
      \emph{moduláris inverz}ének nevezzük modulo $m$.},
    content/english={TODO}
  }
\end{definition}

\begin{exercise}
  \Wrap{
    content/magyar={Írj programot, amely kiszámolja első paraméterének moduláris inverzét modulo
      a második paraméter! Ellenőrzéshez használható az \texttt{inverse\_mod} parancs},
    content/english={TODO}
  }

  \begin{solution}
    \begin{sageexample}
    sage: def modinv(a,m):
    ....:     (d,x,y) = xgcd(a,m)
    ....:     if d == 1:
    ....:         return x % m
    ....:     else:
    ....:         return None
    \end{sageexample}
  \end{solution}
\end{exercise}

\begin{definition}[\Wrap{content/magyar=Lineáris kongruencia-rendszer, content/english=System of congruences}]
  \Wrap {
    content/magyar={Legyen $1<n\in\mathbb{N}$, $a_i,b_i\in\mathbb{Z}$ és $1<m_i\in\mathbb{N}$
      ($1\le i\le n$). Ekkor a \[a_ix\equiv b_i\ (m_i) \quad(1\le i\le n)\] kongruenciák összeségét
      \emph{lineáris kongruencia-rendszer}nek hívunk és csak olyan $x$ egész számot tekintünk
      megoldásnak, amely mindegyiknek külön-külön is megoldása.},
    content/english={TODO}
  }
\end{definition}

\Wrap{
  content/magyar={ A kongruenciarendszerek megoldásának megkereséséhez tekintsünk csak két
    kongruenciát és első lépésként oldjuk meg őket külön-külön. Ezek után a feladat az
    \[ x\equiv c_1 (m_1)\text{ és } x\equiv c_2 (m_2), \] kongruenciarendszer megoldásainak
    megtalálása. A fentieknek megfelelően ez azt jelenti, hogy arra alkalmas $y_1$ és $y_2$
    számokkal \[
      \left.
      \begin{array}{ccc}
        m_1|x-c_1 & \Leftrightarrow & x = c_1 + m_1y_1 \\
        m_2|x-c_2 & \Leftrightarrow & x = c_2 + m_2y_2
      \end{array}
      \right\}\Rightarrow c_1-c_2 = m_1y_1 - m_2y_2,
    \]
    ami tetszőleges $c_1, c_2$ esetén csak akkor lehetséges, ha $m_1$ és $m_2$ relatív prímek.
  },
  content/english={TODO}
}

\Wrap{
  content/magyar={Az általános megoldás megtalálásához az előzőek alapján tegyük fel, hogy
    $(m_1, m_2)=1$ és keressük $x$-et $x=x_1+x_2$ alakban, ahol \[
      \begin{array}{ccccccccc}
        x_1 & \equiv & c_1 & (m_1) & \quad & x_1 & \equiv & 0   & (m_2) \\
        x_2 & \equiv & 0   & (m_1) & \quad & x_2 & \equiv & c_2 & (m_2).
      \end{array}
    \]
    Ebből $x_1$-re $m_1|x_1-c_1$ és $m_2|x_1$, azaz $m_1u_1 = x_1-c_1$ és $m_2v_2 = x_1$, tehát
    ha $m_1u+m_2v=1$, akkor \[
      c_1 = m_1u_1-m_2v_1 = m_1uc_1 + m_2vc_1.
    \]Így $x_1=c_1-m_1uc_1 = m_2vc_1$ és hasonlóan $x_2=c_2-m_2vc_2=m_1uc_2$. Ez alapján azt
    kaptuk, hogy a fenti két kongruenciából álló rendszer egy megoldása \[ x = c_1m_2v + c_2m_1u. \]
  },
  content/english={TODO}
}

\Wrap{
  content/magyar={Az nyilvánvaló, hogy az $x+km_1m_2$ is megoldás lesz tetszőleges $k$ egész
    számra, továbbá a megoldás egyértelmű is modulo $m_1m_2$, mivel bármely két megoldás
    különbsége 0 modulo $m_1$ és $m_2$ is, azaz a megoldások közötti különbség a $[m_1,m_2]$
    többszöröse kell hogy legyen. },
  content/english={TODO}
}

\begin{theorem}[\Wrap{content/magyar=Kínai maradéktétel (KMT), content/english=Chinese remainder theorem (CRT)}]
  \Wrap{
    content/magyar={Legyenek $m_1,m_2,\dots,m_n$ egynél nagyobb páronként relatív prím természetes
      számok. Ekkor az $x\equiv c_i\ (m_i)$ $(1\le i\le n)$ kongruenciarendszernek van megoldása
      és a megoldások kongruensek modulo $m_1m_2\dots m_n$, bármely egész $c_1, c_2,\dots c_n$ egész
      esetén.},
    content/english={TODO}
  }
\end{theorem}

\begin{exercise}
  \Wrap{
    content/magyar={Írj eljárást, amely a kínai maradéktétel megoldását állítja elő. Az
      programnak két lista típusú bemenete legyen, az egyik a kínai maradéktételnél
      szereplő $c$ számok a másik pedig a (páronként relatív prím) modulusok. Ellenőrzéshez
      használható a \texttt{crt} parancs.},
    content/english={TODO}
  }

  \begin{solution}
    \begin{sageexample}
    sage: def chinese(C,M):
    ....:     if len(C) != len(M):
    ....:         return None
    ....:     c, m = C[0], M[0]
    ....:     for i in range(1, len(C)):
    ....:         (g, u, v) = xgcd(m, M[i])
    ....:         if g != 1:
    ....:             return None
    ....:         c = (c*M[i]*v + C[i]*m*u)
    ....:         m *= M[i]
    ....:         c %= m
    ....:     return c
    \end{sageexample}
  \end{solution}
\end{exercise}

\begin{exercise}
  \Wrap{
    content/magyar={Írj eljárást amely lineáris kongruencia-rendszereket old meg! A programnak
      három lista típusu bemenete van: a bal oldalak együtthatóinak, a jobb oldalaknak és a
      modulusoknak listái.},
    content/english={TODO}
  }

  \begin{solution}
    \begin{sageexample}
    sage: def lin_cong_sys(A,B,M):
    ....:     if len(A) != len(B) or len(B) != len(C):
    ....:         return None # should raise some exception
    ....:     c, m = 0, 1
    ....:     for  i in range(len(A)):
    ....:         #sole the ith equation
    ....:         (d, x, _) = xgcd(A[i], M[i])
    ....:         if B[i] % d != 0:
    ....:             return None
    ....:         x = (x*B[i]/d) % M[i]
    ....:         #add to the solution
    ....:         (g, u, v) = xgcd(m, M[i])
    ....:         if (c-x) % g != 0:
    ....:             return None
    ....:         c = (c*M[i]*v + x*m*u)/g
    ....:         m *= M[i]
    ....:         c %= m
    ....:     return c
    \end{sageexample}
  \end{solution}
\end{exercise}

\Wrap{
  content/magyar={A lineáris egyenletek mellett természetesen magasabb rendű és más típusú
    egyenletek is elképzelhetőek. Ezek megoldása általában más problémákat vet fel mint valós
    vagy akár komplex megfelelőjük, de mivel a keresett megoldás egy véges halmazban van, a
    legrosszabb esetben is megkaphatjuk a megoldást végigpróbálva az összes lehetséges értéket.},
  content/english={TODO}
}

\begin{exercise}[\Wrap{content/magyar=Kvardratikus maradékok, content/english=Quadratic residues}]
  \Wrap{
    content/magyar={Írj programot, amely egy adott $m$ esetén megadja azon $b$ $0$ és $m-1$
    közötti számok halmazát, amelyek esetén az \[ x^2 \equiv b\ (m) \] egyenlet megoldható.},
    content/english={TODO}
  }
  \begin{solution}
    \begin{sageexample}
    sage: def quadratic_residues(m):
    ....:     qrs = set()
    ....:     for i in range(m):
    ....:         qrs.add(i^2 % m)
    ....:     return qrs
    \end{sageexample}
  \end{solution}
\end{exercise}

\Wrap{
  content/magyar={Matematikában és az informatikai alkalmazások területén is fontos szerepe van a
    \[ a^x\equiv b\ (m) \] típusú (logaritmushoz hasonló) egyenleteknek. },
  content/english={TODO}
}

\begin{theorem}[(Kis) Fermat-tétel]
  Ha $p$ prím és $a$ tetszőleges egész szám, akkor\[ a^{p-1}\equiv 1\ (p).\]
\end{theorem}

\begin{theorem}[Euler-Fermat-tétel]
  Ha $a$ és $m$ relatív prímek, akkor \[ a^{\varphi(m)}\equiv 1\ (m), \] ahol $\varphi$ az
  Euler-féle $\varphi$ függvény.
\end{theorem}

\begin{definition}[RSA asszimetrikus titkosítás]
  Általában egy asszimetrikus titkosítási sémánál két kulcs áll rendelkezésre (egy publikus és egy
  privát) és a két kucsot egymás után használva visszakapjuk az üzenetet. \emph{RSA} séma esetén
  \begin{itemize}
    \item választunk két elég nagy és megfelelő formájú $p$, $q$ prímet,
    \item egy $e>1$ kitevőt és
    \item számoljuk ki $n=pq$-t, illetve
    \item egy $d$ egész számot, melyre $ed\equiv 1\ (\varphi(n)=(p-1)(q-1))$.
  \end{itemize}
  A publikus kulcs $(n,e)$, a privát kulcs $(n,d)$ lesz és egy $m<n$ szám mint üzenet titkosított
  formáját kapjuk az $s = m^d \mod n$ kiszámolásával. A visszafejtés az Euler-Fermat-tétel
  használatával \[ s^d\equiv (m^d)^e\equiv m^{ed}\equiv m^{\varphi(m)q+1}\equiv m\ (n). \]
\end{definition}

\begin{exercise}
  \Wrap{
    content/magyar={Írj osztályt, amely addott publikus paraméterek esetén megvalósítja a RSA sémát!},
    content/english={TODO}
  }

  \begin{solution}
    \begin{sageexample}
    sage: class RSA(object):
    ....:     #this is just a dummy implementation, not for actual use
    ....:     def __init__(self, length):
    ....:         # uniformly choosen prime is not a good idea in real life
    ....:         p = random_prime(2^(length-2), lbound=2^(length-3))
    ....:         q = random_prime(2^(length+2), lbound=2^(length+1))
    ....:         self.__n = p*q
    ....:         self.__phin = (p-1)*(q-1)
    ....:         self.__e = 3 #should choose this more carefully
    ....:         while gcd(self.__e, self.__phin) != 1:
    ....:             self.__e += 2
    ....:         self.__d = inverse_mod(self.__e, self.__phin)
    ....:     def public_key(self):
    ....:         return (self.__n, self.__e)
    ....:     @staticmethod
    ....:     def encrypt(pubkey, message):
    ....:         return power_mod(message, pubkey[1], pubkey[0])
    ....:     def decrypt(self, secret):
    ....:         return power_mod(secret, self.__d, self.__n)
    \end{sageexample}
  \end{solution}
\end{exercise}

\begin{definition}[Diszkrét logarimus probléma]
  Vegyünk egy $p$ prímet és egy olyan $g$ számot, amely hatványaival modulo $p$ előállítja az
  összes $p$-nél kisebb pozitív számot. Ekkor egy $a$ esetén a $g^a\mod p$ értékből $a$
  meghatározását diszkrét logaritmus problémának hívjuk.
\end{definition}

\begin{definition}[Diffie-Hellman kulcscsere]
  A diszkrét logaritmus problémánál használt $p$ és $g$ publikus paramétereket használva két
  kommunikációs fél (Alice és Bob) tud közös értékben (kulcs) megállapodni Diffie-Hellman
  sémát használva. A séma során mindkét fél választ egy-egy véletlen értéket (titok) és számolják a
  $g^a$ és $g^b$ publikus értékeket. Ezek alapján mindketten ki tudják számolni a közös kulcsot:
  \[g^{ab} = (g^a)^b = (g^b)^a.\]
\end{definition}

\begin{exercise}
  \Wrap{
    content/magyar={Írj programot, amely egy Diffie-Hellman kulcscsere folyamatát szemlélteti.},
    content/english={TODO}
  }

  \begin{solution}
    \begin{sageexample}
    sage: class DH_participant(object):
    ....:     def __init__(self, p, g):
    ....:         self.__p = p
    ....:         self.__g = g
    ....:         self.__x = randint(1, p-1)
    ....:     def get_pub(self):
    ....:         return power_mod(self.__g, self.__x, self.__p)
    ....:     def calculate_common_key(self, pub_of_other):
    ....:         return power_mod(pub_of_other, self.__x, self.__p)
    ....: Alice = DH_participant(65537, 2)
    ....: Bob   = DH_participant(65537, 2)
    ....: #common key
    ....: Alice.calculate_common_key(Bob.get_pub())
    ....: Bob.calculate_common_key(Alice.get_pub())
    \end{sageexample}
  \end{solution}
\end{exercise}

% TODO put the cryptograhpy part in the coding theory as separate section

\end{Section}

\begin{Section}{
  title/english=Polynomials,
  title/magyar=Polinomok}

  
\begin{definition}[\Wrap{content/magyar=Polinom, content/english=Polynomial}]
  \Wrap{
    content/magyar={
      Legyen $R$ egy olyan struktúra, amelyen van értelmezve egy additív és egy
      multiplikatív művelet (például egész számok vagy egy maradékrendszer). Ekkor az $f_i\in R$
      $(i\in\mathbb{N})$ elemekkel, mint együtthatókkal az \[f=(f_0,f_1,\dots)\] sorozatot
      polinomnak nevezzük, ha véges sok eleme nem a nullelem.},
    content/english={TODO}
  }
\end{definition}

\Wrap{
  content/magyar={
    Egy adott struktúra feletti polinomokhoz rendelhetünk változót is, amely segít a polinomok
    kezelésében és jelöli az adott polinomhoz tartozó struktúrát is. Például $x$ jelölheti az egész
    számok fölötti polinomok változóját és ekkor az előző definicióban szereplő $f$ polinom írható
    az \[f = f(x) = f_0 + f_1 x + f_2 x^2 + \dots +f_n = \sum_{i=0}^{n} f_i x^i, \]
    ha minden $n$-nél nagyobb indexű együttható a nullelem. },
  content/english={TODO}
}

\begin{definition}[\Wrap{content/magyar=Fokszám, content/english=Degree}]
  \Wrap{
    content/magyar={
      Egy $f$ polinom esetén a legnagyobb olyan $n\in\mathbb{N}$ indexet, amelyre $f_n$ nem
      nulla \emph{fokszám}nak hívjuk. Ha nincs ilyen, azaz a poinom csak nulla elemet tartalmaz
      (nulla poinom), akkor a fokszám legyen $-\infty$. Jelölése: $\deg(f)$.},
    content/english={TODO}
  }
\end{definition}

\begin{definition}[\Wrap{content/magyar=Műveletek polinomokkal, content/english=Arithmetic of polynomials}]

  \Wrap{
    content/magyar={
      Legyen $f$ az $f_i$ és $g$ a $g_i$ $(i\in\mathbb{N})$együtthatókkal érelmezett polinomok. Ekkor
      a struktúrán értelmezett műveletek segítségével a polinomk fölött is értelmezhetünk aritmetikai
      műveleteket:
      \begin{itemize}
        \item \emph{összeadás} elemenként történik, tehát
          \[(f+g)_i = f_i + g_i\]
          és az eredmény fokszámára $\deg(f+g)\le \max\{\deg(f), \deg(g)\}$
        \item \emph{szorzás}nál minend együtthatót minden együtthatóval össze kell szorozni és az
          eredményt az együtthatók összegével megfelelő indexhez kell adni, azaz
          \[ (fg)_i=\sum_{i=j+k}f_j g_k = \sum_{j=0}^i f_j g_{i-j}\]
          és az eredmény fokszámára $\deg(fg)=\deg(f)+\deg(g)$ (ha nincs olyan két elem $R$-ben,
          melyek szorzata nulla).
      \end{itemize}
    },
    content/english={TODO}
  }
\end{definition}

\begin{exercise}
  Írj polinom osztályt, ahol definiálva van a fokszám és az aritmetika!

  \begin{solution}
    \begin{sageexample}
      sage: class my_poly(object):
      ....:     def __init__(self, F):
      ....:         self.__coeffs = F
      ....:     def deg(self):
      ....:         d = len(self.__coeffs)-1
      ....:         while d >= 0 and self.__coeffs[d] == 0:
      ....:             d -= 1
      ....:         if d >= 0:
      ....:             return d
      ....:         return -Infinit
      ....:     def __add__(self, other):
      ....:         L, i = [], 0
      ....:         while i < len(self.__coeffs) and i < len(other.__coeffs):
      ....:             L.append(self.__coeffs[i] + other.__coeffs[i])
      ....:             i += 1
      ....:         while i < len(self.__coeffs):
      ....:             L.append(self.__coeffs[i])
      ....:             i += 1
      ....:         while i < len(other.__coeffs):
      ....:             L.append(other.__coeffs[i])
      ....:             i += 1
      ....:         return my_poly(L)
      ....:     def __mul__(self, other):
      ....:         L, i = [], 0
      ....:         # as the definition goes
      ....:         for i in range(len(self.__coeffs) + len(other.__coeffs)-1):
      ....:             L.append(0)
      ....:             j = max(0, i-len(self.__coeffs)+2)
      ....:             while j <= i and j < len(self.__coeffs):
      ....:                 L[i] += self.__coeffs[j]*other.__coeffs[i-j]
      ....:                 j += 1
      ....:         return my_poly(L)
      ....:     def __repr__(self):
      ....:         return str(self.__coeffs)
    \end{sageexample}
  \end{solution}
\end{exercise}

\begin{definition}[\Wrap{content/magyar=Polinomfüggvény, content/english=Polynomial function}]
  \Wrap{
    content/magyar={Egy $R$ fölött értelmezett polinomfüggvényen az $\widehat{f}:C\to C$
    leképezést értjük, ha $R\subseteq C$ és $\widehat{f}(c)=f(c)$ a polinom kiértékelése $c$
    helyen.}
  }
\end{definition}

\begin{definition}[\Wrap{content/magyar=Horner-elrendezés, content/english=Horner method}]
    \Wrap{
      content/magyar={Egy $n$-edfokú polinom defnició szerinti kiértékelése $n-1$ összeadással és
        $n(n+1)/2$ szorzással jár. A szorzások számára ennél jóval jobb ($n-1$ db) eljárást kapunk a
        \emph{Horner elrendezés}t haszálva: \[ f(x) = \sum_{i=0}^n f_i x^i =
        f_0 + x(f_1 + x(f_2 + \dots + x(f_{n-1} + xf_n)\dots)). \]},
      content/english={TODO}
    }
\end{definition}

\begin{exercise}
  \Wrap{
    content/magyar={
      Egészítsd ki az előző feladatban adott polinomosztályt egy kiértékelő függvényargumentummal, ami
      a kiértékelést Horner-elrendezésnek megfelelően készíti el!},
    content/english={TODO}
  }

  \begin{solution}
    \begin{sageexample}
      sage: class my_poly2(my_poly):
      ....:     def eval(self, x):
      ....:         y = 0
      ....:         for c in self._my_poly__coeffs:
      ....:             y = x*y + c
      ....:         return y
    \end{sageexample}
  \end{solution}
\end{exercise}

\Wrap{
  content/magyar={
    Egy polinom és annak kiértékelési helyei között szoros kapcsolat áll. Nyilvánvalóan egy polinom
    egyértelmüen meghatározza, hogy milyen értéket vesz az fel egy adott helyen. Kevésbé nyilvánvaló,
    hogy egy $n$-edfokú polinomot egyértelműen meghatároz annak $n+1$ különböző helyen felvett értéke.
    Legyenek ezek a különböző helyek $x_i$-vel  és a felvett értékek $y_i=f(x_i)$-vel jelöve
    $(0\le i \le n)$. Ha sikerülne minden $x_i$ helyhez külön-külön egy-egy olyan $n$-edfokú
    $p_i()$ polinomot konstruálni, amely az $i$-edik helyen $y_i$ értéket a többi $x_j$ $(j\neq i)$
    helyen pedig nullát vesz fel akkor \[f(x) = \sum_{i=0}^{n} p_i(x).\] Egy polinom akkor vesz
    fel egy adott $x_j$ helyen nullát, ha az felírható a $p_i(x) = (x-x_j)r_{ij}(x)$ alakban arra
    alkalmas $r_{ij}$ polinommal. Ez alapján a \[ \prod_{i\neq j=0}^n (x-x_j) \] polinom minden
    $x_i$-től különböző helyen nullát vesz fel. Ahhoz hogy $x_i$ helyen $y_i$-t vegyen fel osszuk
    el a jelenleg felvett értékével $x_i$ helyen majd szorozzuk $y_i$-vel, azaz
    \[ p_i(x) = y_i\frac{\prod_{i\neq j=0}^n (x-x_j)}{\prod_{i\neq j=0}^n (x_i-x_j)} =
      y_i\prod_{i\neq j=0}^n\frac{x-x_j}{x_i-x_j}.\]
  },
  %TODO add here a good example with drawings
  content/english={TODO}
}

\begin{theorem}[\Wrap{content/magyar=Lagrange-interpoláció, content/english=Lagrange interpolation}]
  Egy $f$ $n$-edfokú polinomot egyértelműen meghatároz annak $n+1$ páronként különböző helyen
  felvett értéke és ha $(x_i, y_i)\ (0\le i\le n)$ a hely-érték párok, akkor a polinom felírható a
  \[
    f(x) = \sum_{i=0}^n y_i\prod_{i\neq j=0}^n\frac{x-x_j}{x_i-x_j}
  \] alakban.
\end{theorem}

\begin{exercise}
    \Wrap{
      content/magyar={Írj programot, amely megvalósítja a Lagrange-interpolációt egész számokra,
        azaz egy $n$ elemű egész számokból alkotott párokból ($x_i,y_i$) álló lista esetén
        visszaadja az egész együtthatós interpolációs polinomot (ha van ilyen az egész számok
        felett)! Ellenőrzéshez használható a \[\mathtt{PolynomialRing(QQ).lagrange\_polynomial(L)}\]
        függvény.},
      content/english={TODO}
    }

    \begin{solution}
      TODO
    \end{solution}
\end{exercise}

\begin{definition}[\Wrap{content/magyar=Shamir-féle titokmegosztás, content/english=Shamir's secret sharing}]
  \Wrap{
    content/magyar={Az $S$ titok és $D=\{S_1, S_2, \dots, S_n\}$ látszólag véletlen és látszólag
      $S$-től független adat megfelel a \emph{Shamir-féle titokmegosztás}i sémának $(n,k)$
      paraméterekkel, ha
      \begin{itemize}
        \item Bármely legfeljebb $k-1$ elemű részhalmaza $D$-nek alkalmatlan $S$-re vonatkozó
          információ megszerzésére.
        \item Bármely legalább $k$ elemű részhalmaza $D$-nek alkalmas $S$ helyreállítására.
      \end{itemize}
    },
    content/english={TODO}
  }
\end{definition}

\begin{exercise}
  \Wrap{
    content/magyar={
      Lagrange-interpoláció segítségével valósítsd meg a Shamir-féle titokmegosztást! (Segítség:
        Titok lehet egy amúgy véletlen $n-1$-edfokú polinom konstans tagja.)
    },
    content/english={TODO}
  }

  \begin{solution}
    TODO
  \end{solution}
\end{exercise}

\end{Section}
