
\begin{definition}[\Wrap{content/magyar=Polinom, content/english=Polynomial}]
  \Wrap{
    content/magyar={
      Legyen $R$ egy olyan struktúra, amelyen van értelmezve egy additív és egy
      multiplikatív művelet (például egész számok vagy egy maradékrendszer). Ekkor az $f_i\in R$
      $(i\in\mathbb{N})$ elemekkel, mint együtthatókkal az \[f=(f_0,f_1,\dots)\] sorozatot
      polinomnak nevezzük, ha véges sok eleme nem a nullelem.},
    content/english={TODO}
  }
\end{definition}

\Wrap{
  content/magyar={
    Egy adott struktúra feletti polinomokhoz rendelhetünk változót is, amely segít a polinomok
    kezelésében és jelöli az adott polinomhoz tartozó struktúrát is. Például $x$ jelölheti az egész
    számok fölötti polinomok változóját és ekkor az előző definicióban szereplő $f$ polinom írható
    az \[f = f(x) = f_0 + f_1 x + f_2 x^2 + \dots +f_n = \sum_{i=0}^{n} f_i x^i, \]
    ha minden $n$-nél nagyobb indexű együttható a nullelem. },
  content/english={TODO}
}

\begin{definition}[\Wrap{content/magyar=Fokszám, content/english=Degree}]
  \Wrap{
    content/magyar={
      Egy $f$ polinom esetén a legnagyobb olyan $n\in\mathbb{N}$ indexet, amelyre $f_n$ nem
      nulla \emph{fokszám}nak hívjuk. Ha nincs ilyen, azaz a poinom csak nulla elemet tartalmaz
      (nulla poinom), akkor a fokszám legyen $-\infty$. Jelölése: $\deg(f)$.},
    content/english={TODO}
  }
\end{definition}

\begin{definition}[\Wrap{content/magyar=Műveletek polinomokkal, content/english=Arithmetic of polynomials}]

  \Wrap{
    content/magyar={
      Legyen $f$ az $f_i$ és $g$ a $g_i$ $(i\in\mathbb{N})$együtthatókkal érelmezett polinomok. Ekkor
      a struktúrán értelmezett műveletek segítségével a polinomk fölött is értelmezhetünk aritmetikai
      műveleteket:
      \begin{itemize}
        \item \emph{összeadás} elemenként történik, tehát
          \[(f+g)_i = f_i + g_i\]
          és az eredmény fokszámára $\deg(f+g)\le \max\{\deg(f), \deg(g)\}$
        \item \emph{szorzás}nál minend együtthatót minden együtthatóval össze kell szorozni és az
          eredményt az együtthatók összegével megfelelő indexhez kell adni, azaz
          \[ (fg)_i=\sum_{i=j+k}f_j g_k = \sum_{j=0}^i f_j g_{i-j}\]
          és az eredmény fokszámára $\deg(fg)=\deg(f)+\deg(g)$ (ha nincs olyan két elem $R$-ben,
          melyek szorzata nulla).
      \end{itemize}
    },
    content/english={TODO}
  }
\end{definition}

\begin{exercise}
  Írj polinom osztályt, ahol definiálva van a fokszám és az aritmetika!

  \begin{solution}
    \begin{sageexample}
      sage: class my_poly(object):
      ....:     def __init__(self, F):
      ....:         self.__coeffs = F
      ....:     def deg(self):
      ....:         d = len(self.__coeffs)-1
      ....:         while d >= 0 and self.__coeffs[d] == 0:
      ....:             d -= 1
      ....:         if d >= 0:
      ....:             return d
      ....:         return -Infinit
      ....:     def __add__(self, other):
      ....:         L, i = [], 0
      ....:         while i < len(self.__coeffs) and i < len(other.__coeffs):
      ....:             L.append(self.__coeffs[i] + other.__coeffs[i])
      ....:             i += 1
      ....:         while i < len(self.__coeffs):
      ....:             L.append(self.__coeffs[i])
      ....:             i += 1
      ....:         while i < len(other.__coeffs):
      ....:             L.append(other.__coeffs[i])
      ....:             i += 1
      ....:         return my_poly(L)
      ....:     def __mul__(self, other):
      ....:         L, i = [], 0
      ....:         # as the definition goes
      ....:         for i in range(len(self.__coeffs) + len(other.__coeffs)-1):
      ....:             L.append(0)
      ....:             j = max(0, i-len(self.__coeffs)+2)
      ....:             while j <= i and j < len(self.__coeffs):
      ....:                 L[i] += self.__coeffs[j]*other.__coeffs[i-j]
      ....:                 j += 1
      ....:         return my_poly(L)
      ....:     def __repr__(self):
      ....:         return str(self.__coeffs)
    \end{sageexample}
  \end{solution}
\end{exercise}

\begin{definition}[\Wrap{content/magyar=Polinomfüggvény, content/english=Polynomial function}]
  \Wrap{
    content/magyar={Egy $R$ fölött értelmezett polinomfüggvényen az $\widehat{f}:C\to C$
    leképezést értjük, ha $R\subseteq C$ és $\widehat{f}(c)=f(c)$ a polinom kiértékelése $c$
    helyen.}
  }
\end{definition}

\begin{definition}[\Wrap{content/magyar=Horner-elrendezés, content/english=Horner method}]
    \Wrap{
      content/magyar={Egy $n$-edfokú polinom defnició szerinti kiértékelése $n-1$ összeadással és
        $n(n+1)/2$ szorzással jár. A szorzások számára ennél jóval jobb ($n-1$ db) eljárást kapunk a
        \emph{Horner elrendezés}t haszálva: \[ f(x) = \sum_{i=0}^n f_i x^i =
        f_0 + x(f_1 + x(f_2 + \dots + x(f_{n-1} + xf_n)\dots)). \]},
      content/english={TODO}
    }
\end{definition}

\begin{exercise}
  \Wrap{
    content/magyar={
      Egészítsd ki az előző feladatban adott polinomosztályt egy kiértékelő függvényargumentummal, ami
      a kiértékelést Horner-elrendezésnek megfelelően készíti el!},
    content/english={TODO}
  }

  \begin{solution}
    \begin{sageexample}
      sage: class my_poly2(my_poly):
      ....:     def eval(self, x):
      ....:         y = 0
      ....:         for c in self._my_poly__coeffs:
      ....:             y = x*y + c
      ....:         return y
    \end{sageexample}
  \end{solution}
\end{exercise}

\Wrap{
  content/magyar={
    Egy polinom és annak kiértékelési helyei között szoros kapcsolat áll. Nyilvánvalóan egy polinom
    egyértelmüen meghatározza, hogy milyen értéket vesz az fel egy adott helyen. Kevésbé nyilvánvaló,
    hogy egy $n$-edfokú polinomot egyértelműen meghatároz annak $n+1$ különböző helyen felvett értéke.
    Legyenek ezek a különböző helyek $x_i$-vel  és a felvett értékek $y_i=f(x_i)$-vel jelöve
    $(0\le i \le n)$. Ha sikerülne minden $x_i$ helyhez külön-külön egy-egy olyan $n$-edfokú
    $p_i()$ polinomot konstruálni, amely az $i$-edik helyen $y_i$ értéket a többi $x_j$ $(j\neq i)$
    helyen pedig nullát vesz fel akkor \[f(x) = \sum_{i=0}^{n} p_i(x).\] Egy polinom akkor vesz
    fel egy adott $x_j$ helyen nullát, ha az felírható a $p_i(x) = (x-x_j)r_{ij}(x)$ alakban arra
    alkalmas $r_{ij}$ polinommal. Ez alapján a \[ \prod_{i\neq j=0}^n (x-x_j) \] polinom minden
    $x_i$-től különböző helyen nullát vesz fel. Ahhoz hogy $x_i$ helyen $y_i$-t vegyen fel osszuk
    el a jelenleg felvett értékével $x_i$ helyen majd szorozzuk $y_i$-vel, azaz
    \[ p_i(x) = y_i\frac{\prod_{i\neq j=0}^n (x-x_j)}{\prod_{i\neq j=0}^n (x_i-x_j)} =
      y_i\prod_{i\neq j=0}^n\frac{x-x_j}{x_i-x_j}.\]
  },
  %TODO add here a good example with drawings
  content/english={TODO}
}

\begin{theorem}[\Wrap{content/magyar=Lagrange-interpoláció, content/english=Lagrange interpolation}]
  Egy $f$ $n$-edfokú polinomot egyértelműen meghatároz annak $n+1$ páronként különböző helyen
  felvett értéke és ha $(x_i, y_i)\ (0\le i\le n)$ a hely-érték párok, akkor a polinom felírható a
  \[
    f(x) = \sum_{i=0}^n y_i\prod_{i\neq j=0}^n\frac{x-x_j}{x_i-x_j}
  \] alakban.
\end{theorem}

\begin{exercise}
    \Wrap{
      content/magyar={Írj programot, amely megvalósítja a Lagrange-interpolációt egész számokra,
        azaz egy $n$ elemű egész számokból alkotott párokból ($x_i,y_i$) álló lista esetén
        visszaadja az egész együtthatós interpolációs polinomot (ha van ilyen az egész számok
        felett)! Ellenőrzéshez használható a \[\mathtt{PolynomialRing(QQ).lagrange\_polynomial(L)}\]
        függvény.},
      content/english={TODO}
    }

    \begin{solution}
      TODO
    \end{solution}
\end{exercise}

\begin{definition}[\Wrap{content/magyar=Shamir-féle titokmegosztás, content/english=Shamir's secret sharing}]
  \Wrap{
    content/magyar={Az $S$ titok és $D=\{S_1, S_2, \dots, S_n\}$ látszólag véletlen és látszólag
      $S$-től független adat megfelel a \emph{Shamir-féle titokmegosztás}i sémának $(n,k)$
      paraméterekkel, ha
      \begin{itemize}
        \item Bármely legfeljebb $k-1$ elemű részhalmaza $D$-nek alkalmatlan $S$-re vonatkozó
          információ megszerzésére.
        \item Bármely legalább $k$ elemű részhalmaza $D$-nek alkalmas $S$ helyreállítására.
      \end{itemize}
    },
    content/english={TODO}
  }
\end{definition}

\begin{exercise}
  \Wrap{
    content/magyar={
      Lagrange-interpoláció segítségével valósítsd meg a Shamir-féle titokmegosztást! (Segítség:
        Titok lehet egy amúgy véletlen $n-1$-edfokú polinom konstans tagja.)
    },
    content/english={TODO}
  }

  \begin{solution}
    TODO
  \end{solution}
\end{exercise}