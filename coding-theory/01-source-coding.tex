\Wrap{
  content/magyar={
    A forráskódolás a kódok hosszával foglalkozik, vagyis azzal a kérdéssel, hogy az adott
    mennyiségű információt mekkora mennyiségű adattal tudjuk tárolni. Ehhez szükségünk van az
    információ alapegységére $r$, amely bináris esetben 2.
  },
  content/english={TODO}
}

\begin{definition}[\Wrap{content/magyar=Entrópia, content/english=Entropy}]
  \Wrap{
    content/magyar={
      A kódolt és kódolatlan üzenetek ($m$) karakterekre bonthatóak, melyek önmagukban is, de
      leginkább az üzenetben elfoglalt helyük segítségével információt tárolnak. Arra hogy mennyi
      információt hordoz egy üzenet a karakterek rendezetlensége utal. Például egy egyetlen
      karaktert ismételgető forrás által küldött forrás információmennyisége kisebb mint egy olyané
      ami a karaktereket valamilyen bonyolutabb szabály szerint fűzi egymás után (szöveg). Erre a
      rendezetlenségre és így a relatív információmennyiségre utal az \emph{entrópia}, amely ha az
      egyes karakterek előfordulásának valószínűsége $p_1,p_2,\dots,p_n$ $m$-ben, akkor értéke
      \[ H_r(m) = -\sum_{i=1}^n p_i\log_r p_i. \]},
    content/english={TODO}
  }
\end{definition}

\Wrap{
  content/magyar={Az entrópia értéke akkor a legkisebb ($0$), ha az üzenet csak egy karaktert
    tartalmaz; és akkor a legnagyobb ($\log_r n$), ha minden üzenet azonos valószínüséggel szerepel.
    Ebből közvetlenül tudunk következtetni, hogy a szöveg mennyire ,,tömör'', mivel az
    entrópia megadja hogy a karaktereket mennyire ,,jól'' alkalmazzuk.
 },
  content/english={TODO}
}

\Wrap{
  content/magyar={A kódolás során legfontosabb szempont a dekódolhatóság. Egy kódhalmaz
    (kódszavakat tartalmazó halmaz) azon tulajdonsága, hogy bármely belőle készített kódolt
    egyértelműen dekódolható-e azonban nem minden esetben könnyű, így szokás a kódhalmazra (kódra)
    vonatkozó következő fogalmakat definiálni.},
  content/english={TODO}
}

\begin{definition}[\Wrap{content/magyar={Felbontható, egyenletes, vesszős és prefix kód}, content/english={TODO}}]
  \Wrap{
    content/magyar={Legyen a kódszavak ábécéje $B$ és $\alpha, \beta, \gamma\in B*$ az ábécé
      feletti szavak (nem feltétlenül kódszavak). A kód ekkor},
    content/english={TODO}
  }
  \begin{itemize}
    \item\Wrap{
      content/magyar={\emph{felbontható}, ha bármely szöveg egyértelműen dekódolható;},
      content/english={TODO}}
    \item\Wrap{
      content/magyar={\emph{egyenletes}, ha minden kódszó azonos számú karaktert tartalmaz;},
      content/english={TODO}
    }
    \item\Wrap{
    content/magyar={\emph{vesszős}, ha minden kódszó felírható az $\alpha\gamma$ alakban és
      ha $\alpha\gamma\beta$ kódszó, akkor a $\beta = \varepsilon$, ahol $\varepsilon$ az üres szó
      és $\gamma\neq\varepsilon$;},
    content/english={TODO}
    }
    \item\Wrap{
    content/magyar={\emph{prefix}, ha a kódszavak halmaza prefixmentes, azaz ha az
      $\alpha\neq\varepsilon$és $\alpha\beta$ is kódszó, akkor $\beta=\varepsilon$;},
    content/english={TODO}
    }
  \end{itemize}

\end{definition}

\begin{definition}[\Wrap{content/magyar=Betűnkénti kódolás, content/english={TODO}}]
  A kódolás betűnként történik, ha a szöveg $A$ ábécéje és a kódszavak $B$ ábécéje között létezik
  egy $\varphi\in A\to B$ injektív (minden értéket felvesz pontosan egyszer) leképezés.
\end{definition}

% TODO remove this when there is other type of source coding introduced in the document
\Wrap{
  content/magyar={A továbbiakban csak betűnkénti kódolásról fogunk beszélni.},
  content/english={TODO}
}
