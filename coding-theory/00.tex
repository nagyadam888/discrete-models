% TODO write introduction to coding theory

\Wrap{
  content/english={TODO},
  content/magyar ={
    A kódoláselmélet kódok tulajdonságait vizsgálja abból a szempontból, hogy mennyire
    felelnek meg a különféle alkalmazási területeken. Ezen vizsgálat során felmerülő feladatok
    általánosíthatók a következő modellel: egy küldő egy vagy több üzenetet probál meg eljuttatni
    egy fogadóhoz valamilyen tulajdonságokkal rendelkező csatornán keresztül.}
}

\begin{Section}{
  title/english=Source coding,
  title/magyar=Forráskódolás}

  \Wrap{
  content/magyar={
    A forráskódolás a kódok hosszával foglalkozik, vagyis azzal a kérdéssel, hogy az adott
    mennyiségű információt mekkora mennyiségű adattal tudjuk tárolni. Ehhez szükségünk van az
    információ alapegységére $r$, amely bináris esetben 2.
  },
  content/english={TODO}
}

\begin{definition}[\Wrap{content/magyar=Entrópia, content/english=Entropy}]
  \Wrap{
    content/magyar={
      A kódolt és kódolatlan üzenetek ($m$) karakterekre bonthatóak, melyek önmagukban is, de
      leginkább az üzenetben elfoglalt helyük segítségével információt tárolnak. Arra hogy mennyi
      információt hordoz egy üzenet a karakterek rendezetlensége utal. Például egy egyetlen
      karaktert ismételgető forrás által küldött forrás információmennyisége kisebb mint egy olyané
      ami a karaktereket valamilyen bonyolutabb szabály szerint fűzi egymás után (szöveg). Erre a
      rendezetlenségre és így a relatív információmennyiségre utal az \emph{entrópia}, amely ha az
      egyes karakterek előfordulásának valószínűsége $p_1,p_2,\dots,p_n$ $m$-ben, akkor értéke
      \[ H_r(m) = -\sum_{i=1}^n p_i\log_r p_i. \]},
    content/english={TODO}
  }
\end{definition}

\Wrap{
  content/magyar={Az entrópia értéke akkor a legkisebb ($0$), ha az üzenet csak egy karaktert
    tartalmaz; és akkor a legnagyobb ($\log_r n$), ha minden üzenet azonos valószínüséggel szerepel.
    Ebből közvetlenül tudunk következtetni, hogy a szöveg mennyire ,,tömör'', mivel az
    entrópia megadja hogy a karaktereket mennyire ,,jól'' alkalmazzuk.
 },
  content/english={TODO}
}

\Wrap{
  content/magyar={A kódolás során legfontosabb szempont a dekódolhatóság. Egy kódhalmaz
    (kódszavakat tartalmazó halmaz) azon tulajdonsága, hogy bármely belőle készített kódolt
    egyértelműen dekódolható-e azonban nem minden esetben könnyű, így szokás a kódhalmazra (kódra)
    vonatkozó következő fogalmakat definiálni.},
  content/english={TODO}
}

\begin{definition}[\Wrap{content/magyar={Felbontható, egyenletes, vesszős és prefix kód}, content/english={TODO}}]
  \Wrap{
    content/magyar={Legyen a kódszavak ábécéje $B$ és $\alpha, \beta, \gamma\in B*$ az ábécé
      feletti szavak (nem feltétlenül kódszavak). A kód ekkor},
    content/english={TODO}
  }
  \begin{itemize}
    \item\Wrap{
      content/magyar={\emph{felbontható}, ha bármely szöveg egyértelműen dekódolható;},
      content/english={TODO}}
    \item\Wrap{
      content/magyar={\emph{egyenletes}, ha minden kódszó azonos számú karaktert tartalmaz;},
      content/english={TODO}
    }
    \item\Wrap{
    content/magyar={\emph{vesszős}, ha minden kódszó felírható az $\alpha\gamma$ alakban és
      ha $\alpha\gamma\beta$ kódszó, akkor a $\beta = \varepsilon$, ahol $\varepsilon$ az üres szó
      és $\gamma\neq\varepsilon$;},
    content/english={TODO}
    }
    \item\Wrap{
    content/magyar={\emph{prefix}, ha a kódszavak halmaza prefixmentes, azaz ha az
      $\alpha\neq\varepsilon$és $\alpha\beta$ is kódszó, akkor $\beta=\varepsilon$;},
    content/english={TODO}
    }
  \end{itemize}

\end{definition}

\begin{definition}[\Wrap{content/magyar=Betűnkénti kódolás, content/english={TODO}}]
  A kódolás betűnként történik, ha a szöveg $A$ ábécéje és a kódszavak $B$ ábécéje között létezik
  egy $\varphi\in A\to B$ injektív (minden értéket felvesz pontosan egyszer) leképezés.
\end{definition}

% TODO remove this when there is other type of source coding introduced in the document
\Wrap{
  content/magyar={A továbbiakban csak betűnkénti kódolásról fogunk beszélni.},
  content/english={TODO}
}


\end{Section}

\begin{Section}{
  title/english=Channel coding,
  title/magyar=Hibajelző és hibajavító kódolás}

  \Wrap{
  content/magyar={A kódolási feladatok esetén a kódolt üzenetnek egy csatornán keresztül kell
    eljutnia a fogadóhoz. Ez a csatorna lehet zajmentes, azaz garantált az, hogy amit a küldő a
    csatornába juttatott a fogadó hiba nélkül megkapja. Sajnos a valós alkalmazások esetén nincs így,
    alacsony kommunikációs szinten nem tudjuk vagy nem éri meg garantálni a bithelyes áramlást.
    A megoldás az, hogy olyan ún.~hibakorlátozókódot kunstruálunk, ami képes jelezni és/vagy
    javítani az átvitel közben keletkezett hibát. Az ilyen kódokok konstruálása általában nem
    egyszerű feladat és több tényezőt is figyelembe kell venni, mint például
    \begin{itemize}
        \item hijelző és hibajavító képesség;
        \item mennyivel lesz hosszabb a kódolt adat;
        \item mennyibe ,,kerül'' a kódolás és/vagy a dekódolás;
        \item milyen típusú (pl. csomókban vagy elszórtan) és eloszlású hibára számíthatunk.
    \end{itemize}
    },
  content/english={TODO}
}

\begin{definition}[\Wrap{content/magyar={(Pontosan) $t$-hibajelző kód}, content/english=TODO}]
  \Wrap{
    content/magyar={Egy kódot \emph{$t$-hibajelző}nek nevezünk, ha bármely $t$ hibát képes
      jelezni és \emph{pontosan $t$-hibajelző}, ha legfeljebb $t$ hibát tud biztosan észlelni,
      azaz van olyan $t+1$ hiba, amit már nem. },
    content/english={TODO}
  }
\end{definition}

\begin{definition}[\Wrap{content/magyar={(Pontosan) $t$-hibajavító kód}, content/english=TODO}]
  \Wrap{
    content/magyar={Egy kódot \emph{$t$-hibajavító}nek nevezünk, ha bármely $t$ hibát képes
      javítani és \emph{pontosan $t$-hibajavító}, ha legfeljebb $t$ hibát tud biztosan javítani,
      azaz van olyan $t+1$ hiba, amit már nem. },
    content/english={TODO}
  }
\end{definition}

\begin{definition}[\Wrap{content/magyar={Ismétléses-kód}, content/english={TODO}}]
    Talán a legegyszerűbb kódkonstrukció közé tartozsik az \emph{ismétléses-kód}, ami esetén
    minen egyes karaktert $1<k$-szor megismétlünk. Például a $01001$-ből $000111000000111$ lesz, ha
    $k=3$.
\end{definition}

Látható az ismétléses kód pontosan $k-1$ hibát képes jelezni, hiszen ha a $k$ hiba egyetlen
kódolás előtti karakterhez tartozik, akkor azt hibásan fogja dekódolni. A konstrukció hibája
is egyértelmű, $t$ hiba jelzéséhez $t+1$-szer hosszabb kódot készít. t=1 esetén a duplázó kóddal
megegyező hibajelző képeséggel (a definíció szerint) a paritásbites kódolással.

\begin{definition}[Paritásbites kód]
    A paritásbites kódot úgy kapjuk, hogy minden bináris szót kiegészítünk egy bittel annak
    megfelelően, hogy a benne lévő 1-esek száma páros vagy páratlan. Ha a páratlan számú egyesek
    esetén 1-et írunk a szóhoz különben 0-t, akkor \emph{párosra kiegészített paritásbites
    kódolás}t kajuk, míg ha pont fordítva járunk el akkor a \emph{páratlanra} egészítünk ki.
\end{definition}

A paritásbittel való kiegészítés 1 hibát tud észlelni, mivel már két bit változása esetén ismét
érvényes kódszót kapunk.

Észrevehető, hogy a hibajelző képesség attól függ, hogy legalább hány változtatás szükséges ahhoz,
hogy érvényes kódszót kapjunk. Ehhez a kapcsolat pontos kimondásához ad segítséget a következő
definíció.

\begin{definition}[Hamming távolság]
    Egy kód két szava közötti \emph{Hamming távolság}án $d(u,v)$ azon poziciók számát értjük, ahol a
    két szó eltér egymástól. Például $d(0110,1011) = 3$. A teljes kód távolságán a kódszavak
    távolságának minimumát értjük értjük, azaz \[ d(C)= \min\{d(u,v)|u,v\in C\}.\]
\end{definition}

Ha egy kód távolsága $d$, akkor az pontosan $d-1$ hibajavító, hiszen $d-1$ módosítás esetén biztosan
nem kaphatunk érvényes kódszót, de van olyan szópár, amely $d$ módosítással felcserélhetők.

A hibajavító képességhez először meg kell állapodni abban, hogy hogyan szeretnénk javítani a
hibákat, azaz meg kell állapodni abban a leképezésben, ami a nem kódszavakat kódszavakra képezi le.
Ehhez is a távlság fogalmát fogjuk használni.

\begin{definition}[Minimális súlyú dekódolás]
    A \emph{minimális súlyú dekódolás} esetén a fogadott nem kódszavakat úgy dekódoljuk, mintha a
    hozzá legközlebbi érvényes kódszót kaptuk volna, ha van ilyen.
\end{definition}

Minimális súlyú dekódolás esetén a hibajavító képesség már kónnyen megadható a kód távolságának
ismeretében. Ha a kódban csak a távolság felénél kevesebb hiba keletkezett, akkor az még mindig
közelebb lesz az eredeti kódszóhoz mint bármely másikhoz, így helyesen javítunk, azaz a $d$
távolsággal rendelkező kód pontosan $\lfloor\frac{d-1}{2}\rfloor$ hibajaító.

\begin{definition}{Kétdimenziós paritásbites kódolás}
    TODO
\end{definition}



\end{Section}

%\begin{Section}{
%  title/english=Crytography,
%  title/magyar=Kriptogárfia}
%
%  \input{number-theory/03-cryptography.tex}
%
%\end{Section}


