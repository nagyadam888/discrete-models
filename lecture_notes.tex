\documentclass{amsbook}

% LANGUAGE: choose one of the following line and compile the doc
\usepackage[magyar]{babel}
%\usepackage[english]{babel}

\usepackage[languages={english,magyar}]{multilang}
\usepackage{multilang-sect}
\usepackage{lecture_notes_commands}

% set font encoding for PDFLaTeX, XeLaTeX, or LuaTeX
\usepackage{ifxetex,ifluatex}
\newif\ifxetexorluatex
\ifxetex
  \xetexorluatextrue
\else
  \ifluatex
    \xetexorluatextrue
  \else
    \xetexorluatexfalse
  \fi
\fi

\ifxetexorluatex
  \usepackage{fontspec}
\else
  \usepackage[T1]{fontenc}
  \usepackage[utf8]{inputenc}
  \usepackage{lmodern}
\fi

\usepackage{hyperref}
\usepackage{sagetex}
% http://doc.sagemath.org/html/en/tutorial/sagetex.html

% For centering things
\usepackage{chngcntr}

\usepackage{bookmark}

% The counting will restart for every part.
\counterwithin*{section}{part}

% There is no identation at the begining of paragragaphs
\setlength{\parindent}{0pt}

\Title{
  title/english=Applications of Discrete models,
  title/magyar =Diszkrét modellek alkalmazásai
}
\Author{
  author/english=Adam Nagy,
  author/magyar =Nagy Ádám
}

\begin{document}

\maketitle

\setcounter{part}{1}
\begin{Part*}{
    title/english=Number Theory,
    title/magyar =Számelmélet}
  
  \Wrap{
    content/english={
      The \emph{SageMath} application gives the whole structure of integer
      numbers as the \texttt{ZZ} object which behaves according to the "usual"
      mathematical definition.},
    content/magyar ={
      A \emph{SageMath} programcsomagban az egész számokat a \texttt{ZZ}
      objektummal kapjuk meg a ,,szokásos``  matematikai definíciónak
      megfelel\H en.}
  }
  \begin{sageexample}
    sage: type(ZZ)
  \end{sageexample}
  
  \Wrap{
    content/magyar={
      Természetesen nem kell minden esetben használnunk a konstruktort, ha egy
      egész számmal szeretnénk dolgozni, a rendszer automatikusan felismeri.
    },
    content/english={
      Fortunatelly there is no need to use the constructor when we want to
      create an integer, the system automaticly recognizes such numbers. 
    }
  }
  \begin{sageexample}
    sage: a,b = ZZ(4), 4
    sage: type(a) == type(b)
    sage: a == b
  \end{sageexample}

  \Wrap{
    content/magyar={Aritmetiaki m\H uveletek a ,,szokásosak``:},
    content/english={Arithmetic on integers:}
  }
  \begin{itemize}
    \item 
      \Wrap{
        content/magyar={Összeadás, kivonás},
        content/english={Addition, substraction}
      }: \texttt{+,-};
    \item 
      \Wrap{
        content/magyar={Szorzás, hatványozás},
        content/english={Multiplication, raising to a power}
      }: \texttt{*,\^};
    \item 
      \Wrap{
        content/magyar={Egész érték\H u osztás és maradékképzés},
        content/english={Division over the integers and residue}
      }: \texttt{//,\%}.
  \end{itemize}
  
  \Wrap{
    content/magyar={
      \emph{Megjegyzés:} A \texttt{/} m\H uvelet eredménye egy racionális
      szám, sőt valójában a jelenléte elég, hogy innentől racionálisként
      tekintsen a megadott adatokra.},
    content/english={
      \emph{Note:} The result of the \texttt{/} operand is a rational number
      and in general if this operand  is used then the result's type will 
      be rational.}
  }
  \begin{sageexample}
    sage: 2/3
    sage: type(2/3)
    sage: 1/1
    sage: type(1/1)
  \end{sageexample}

\begin{Section}{
    title/english=Divisor,
    title/magyar=Oszthatóság}

  \begin{definition}[\Wrap{content/magyar=Osztó, content/english=Divisor}]
    \Wrap{
      content/magyar={Az $a$ osztója $b$-nek és $b$ többszöröse $a$-nak, azaz $a|b$, ha},
      content/english={$a$ is a divisor of $b$ and $b$ is a multiple of $a$ that is to say
        $a|b$, if}
    }
    \[ \exists c: b = ac.\]
  \end{definition}
    
  \begin{exercise}
    Írd meg azt a függvényt, amely eldönti, hogy az els\H o
    argumentuma osztható-e a másodikkal az alábbi példán kív\H ul még 4 
    különböz\H o módon!
  \end{exercise}
    
  \begin{sageexample}
    sage: def divides0(a,b):
    ....:     return (a/b).is_integer()
    sage: divides0(5,2)
    sage: divides0(6,3)
  \end{sageexample}

  
  Oszthatóság tulajdonságai természetes számok esetén:
  \begin{enumerate}
    \item Részbenrendezés, azaz
      \begin{itemize}
        \item Reflexív ($\forall a\in\mathbb{Z}: a|a$),
        \item Antiszimmetrikus 
          ($\forall a,b\in\mathbb{N}: a|b\wedge b|a\Rightarrow a=b$),
        \item Tranzitív 
          ($\forall a,b,c\in\mathbb{Z}: a|b \wedge b|c \Rightarrow a|c$);
      \end{itemize}
    \item minden szám osztja $0$-t;
    \item $1$ minden számnak osztója;
    \item $0$ csak saját magának osztója;
    \item ha $a|b$ és $c|d$, akkor $ac|bd$;
    \item ha $a|b$, akkor $\forall k\in \mathbb{Z}: ak|bk$;
    \item ha $k\in \mathbb{N}\setminus{0}:ak|bk$, akkor $a|b$;
    \item ha $a|b$ és $a|c$, akkor $a|b+c$;
    \item egy pozitív szám minden osztója kisebb vagy egyenlő mint a szám maga.
  \end{enumerate}

  A részbenrendezések, így az oszthatóság is megadható egy speciális objektummal:
  \texttt{Poset} (\emph{P}artially \emph{O}rdered \emph{S}et).
  Az ilyen objektumot ábrázolva kapjuk a részberendezések szemléltetésére
  használt Hasse-diagramot.
  \begin{sageexample}
    sage: k = 15
    ....: P = Poset((Set([2..k]), lambda a,b: b % a == 0))
  \end{sageexample}
  \begin{figure}[h!]
    \centering
    \sageplot[scale=.6][png]{P.plot(talk=True)}
    \caption{Hasse-diagram oszthatóság esetén $2$ és $k$ közötti természetes
    számokon.  (\texttt{P.plot(talk=True)})}
  \end{figure}

  \begin{exercise} Írj programot, amely egy adott számhalmaz esetén megszámolja hány
    él van az oszthatóság relációhoz tartozó  Hasse-diagramban! Ellen\H orzésre lehet 
    használni az alábbi kódot.
  \end{exercise}
  \begin{sageexample}
    sage: len(P.cover_relations_graph().edges())
  \end{sageexample}

  \begin{exercise} Írj programot, amely egy adott egész szám esetén kiírja
    osztóinak számát, illetve osztóinak összegét! Ellenőrzéshez
    használhatjuk a \verb|sigma(n,0)| és \verb|sigma(n,1)| parancsokat.
  \end{exercise}

  \begin{exercise} Írj programot, amely a természetes számok egy adott
    halmazában megkeresi a tökéletes számokat (tökéletes szám: osztóinak 
    összege megegyezik a számmal, pl. 6).
  \end{exercise}
  
  \begin{exercise}{Aliquot} Természetes számok esetén definiálhatjuk a következő
    sorozatot: $(s_0 = n; s_{i+1} = \sigma(s_i) - s_i)$, ahol a $\sigma(n)$
    az $n$ osztóinak összege. A sorozat vagy terminál nulla értékkel vagy
    periódikussá válik. Készíts programot, amely egy adott természetes szám 
    esetén kiszámolja az említett sorozatot. (Ha nem terminál, akkor csak az 
    első periódust írja ki.)
  \end{exercise}

  \begin{definition}[Asszociált] Az $a\neq b$ elemek asszociáltak, ha $a|b$ és 
    $b|a$ is teljesül.
  \end{definition}

  \begin{definition}[Egység] Egy $e$ elem egységelem, ha bármely $a$ elemre
    $a=ea=ae$. Az egységelem asszociáltjait egységeknek hívjuk.
  \end{definition}

  \begin{definition}[Irreducibilis] Egy nem egység $a$ elemet felbonthatatlannak vagy 
    irreducibilisnek nevezünk, ha $a=bc$ esetén $b$ és $c$ közül az 
    egyik egység.
  \end{definition}

  \begin{definition}[Prím] Egy nem egység $p$ elemet prímnek nevezünk, ha $p|ab$
    esetén a $p|a$ vagy a $p|b$ közül legalább az egyik teljesül.
  \end{definition}

  Megjegyzések:
  \begin{enumerate}
    \item Az egység és egységelem két külön fogalom, egységelem egyedi, 
      amíg az is előfordulhat, hogy a struktúra összes eleme egység (pl.: $\mathbb{Q}$).
    \item Az egység alternatív definíciója: $a$ egység, ha bármely $b$ elem
      felírható $b=ac$ alakban.
    \item Minden prímelem egyben irreducibilis is, hiszen \[
        p = ab \Rightarrow p|ab \Rightarrow p|b \Rightarrow b = qp \Rightarrow
        p = (aq)p \Rightarrow a \text{ egység}.
      \]
    \item Természetes számok esetén (és minden Gauss-gy\H ur\H uben), ha egy
      elem irreducibilis, akkor prím is.
    \item Lehet olyan struktúrát mutatni, ahol van olyan irreducibilis elem,
      ami nem prímelem. Például ha tekintjük az egész konstans taggal rendelkező
      egyváltozós polinomokat, azaz a $\mathbb{Z} + x\mathbb{R}[x]$ struktúrát,
      akkor az $x$ felbonthatatlansága nyilvánvaló; ugyanakkor az
      $x|(x\sqrt{2})^2$ teljesül, de $x$ nem osztja $x\sqrt{2}$-t, ui. az 
      osztás eredményének is benne kellene lennie a struktúrában, de 
      $\sqrt{2}\notin\mathbb{Z}$. 
  \end{enumerate}

  \begin{exercise} A $\mathbb{Z}_m$ struktúra alatt a ${0,1,\dots, m-1}$
    számokat értjük úgy, hogy az összeadás és szorzás műveletet $\mod m$ értjük.
    Írj programot amely egy adott $m$ esetén definíció alapján meghatározza az 
    egységeket, irreducibiliseket és prímeket!
  \end{exercise}

  Természetes számok esetén nyilvánvaló, hogy végtelen sok prím van, hiszen ha
  feltennénk, hogy véges sok van, akkor azokat összeszorozva és az eredményt
  eggyel megnövelve olyan számot kapnánk, aminek egyik sem osztója.
  A prímek számára becslést az $\frac{x}{\ln x}$ formulával kaphatunk, amíg
  \emph{SageMath}-ban a \texttt{prime\_pi(x)} függvénnyel kaphatjuk meg a
  pontos számukat.
  \begin{sageexample}
    sage: P1 = plot(x/log(x), (2, 200), scale='semilogy', \
    ....:     fill=lambda x: prime_pi(x),fillcolor='red')
    ....: P2 = plot(1.13*log(x), (2, 200), \
    ....:     fill=lambda x: nth_prime(x)/floor(x), fillcolor='red')
    ....: P = graphics_array([P1, P2])
  \end{sageexample}
  \begin{figure}[ht]
    \centering
    \sageplot[scale=.6]{P,figsize=[8,4]}
    \caption{Prímek számának és növekedésének becslése (\texttt{P1,P2}) }
  \end{figure}

  \begin{theorem}[Számelmélet alaptétele] Minden pozitív természetes szám a 
    sorrendtől eltekintve egyértelműen felírható prímszámok szorzataként.
  \end{theorem}

  \begin{exercise}[Erasztotenész szitája] Adj programot, amely megadja az 
    összes prímet egy adott számig, azaz ugyanazt az eredmény adja mint a 
    $\mathtt{primes\_first\_n(n)}$!
  \end{exercise}

  \begin{exercise} Írd meg az előző feladatot hatékonyabban úgy, hogy a páros
    számok ne is kerüljenek be a táblába!
  \end{exercise}

  \begin{exercise} Írd meg a prímszitát úgy, hogy a $2,3$ és $5$-tel osztható
    számok ne kerüljenek a táblába! Ehhez a számokat $30i+M[j]$ alakban tárold
    ($30=2\cdot 3\cdot 5$), ahol $i\in[1,\lceil n\rceil]$; $j\in[1,8]$ és
    $M=[1,7,11,13,17,19,23,29]$.
  \end{exercise}

  \begin{exercise}[Ikerprímek]
      Természetes számok esetén az olyan prímeket melyeknek különbsége 2
      ikerprímeknek hívjuk. Írj programot, amely megkeresi az összes ikerprímet
      adott $a$ és $b$ között.
  \end{exercise}

  \begin{definition}[Legnagyobb közös osztó]
    Az $a$ és $b$ legnagyobb közös osztója az a $c = (a,b)$, amelyre
    \[ c|a \wedge c|b \wedge \forall d: d|a \wedge d|b \Rightarrow d|c. \]
    Ha a struktúrában van ,,szokásos`` rendezés (ilyen az egész számok), akkor ezek 
    közül csak a legnagyobbat tekintjük legnagyobb közös osztónak. (Például a 12
    és 18 egész számokra a 6 és -6 is megfelelő lenne, de (12,18)=6.)
  \end{definition}

  \begin{exercise}
    Írj programot, kiszámolja a legnagyobb közös osztót a $\mathtt{factor}$
    parancs segítségével! Tesztelésre használható a $\mathtt{gcd(a,b)}$ parancs.
  \end{exercise}

  \begin{definition}[Legkisebb közös többszörös]
    Az $a$ és $b$ legkisebb közös többszöröse az a $c$, amelyre $c\cdot(a,b)=ab$.
  \end{definition}

  \begin{exercise}
    Írj programot, kiszámolja a legkisebb közös többszöröst a $\mathtt{factor}$
    parancs segítségével (és gcd használata nélkül)! Tesztelésre használható a
    $\mathtt{lcm(a,b)}$ parancs.
  \end{exercise}

  \begin{definition}[Relatív prím]
    Ha $(a,b)=1$, akkor $a$ és $b$ relatív prímek.
  \end{definition}

  \begin{definition}[Euklideszi algoritmus] A legnagyobb közös
    osztója $a$-nak és $b$-nek kiszámolható a következ\H o algoritmussal
    (amennyiben van maradékos osztás a struktúrában):
    \begin{enumerate}
      \item Legyen $a = qb + r$, ahol $0 \le r < b$. 
      \item Ha $r = 0$, akkor a legnagyobb közös osztó $a$.
      \item $b \leftarrow a$
      \item $a \leftarrow  r$
      \item Ugorjunk (1)-re.
    \end{enumerate}
  \end{definition}

  \begin{exercise}
    Készítsd el a fenti algoritmust és hasonlítsd össze a korábbi legnagyobb
    közös osztót számoló program futási idejével!
  \end{exercise}
  
  \begin{exercise}[Binary GCD] Írj programot a legnagyobb közös osztó kiszámolására, ami
    csak additív és shift műveleteket használ (hatékony
    számítógépen) az alábbi összefüggéseket használva!
    \begin{itemize}
      \item $(2a,2b) = 2(a,b)$,
      \item $(2,b)=1 \Rightarrow (2a,b) = (a,b)$,
      \item $(a,b) = (a-b,b)$ és így ha $a$ és $b$ is páratlan, akkor $a-b$
        páros.
    \end{itemize}
  \end{exercise}

  \begin{theorem} 
    Létezik olyan $x$ és $y$, amelyekre \[ ax+by=(a,b). \]
  \end{theorem}
  
  \begin{definition}[B\H ovített Euklideszi algoritmus] 
    Az $(a,b)$ és a hozzá tartozó $x$, $y$ értékek ($(a,b)=ax+by$) meghatározására szolgáló
    algoritmus. A hagyományos algoritmushoz hasonlóan a maradékokat ($r_i$)
    fogjuk számolni az \[r_i = r_{i-2} - q_ir_{i-1}\] alakban, továbbá használjuk
    az \[
      \begin{array}{rcl}
        ax_i+by_i & = & r_i \\
                  & = & r_{i-2} - q_ir_{i-q} \\
                  & = & (ax_{i-2} + bx_{i-2}) - q_i(ax_{i-1} + by_{i-1}) \\
                  & = & a(x_{i-2} - q_ix_{i-1}) + b(y_{i-2} - q_iy_{i-1})
      \end{array}
    \] invariánst. Ennek eleget téve az algoritmus
    \begin{enumerate}
      \item $x_0, y_0, r_0 \leftarrow 1, 0, a$;
      \item $x_1, y_1, r_1 \leftarrow 0, 1, b$;
      \item $i \leftarrow 1$
      \item Ha $r_i = 0$ akkor a megoldás $(x_i,y_i,r_i)$, különben $i\leftarrow i+1$;
      \item $q_i \leftarrow \lfloor r_{i-2}/r_{i-1}\rfloor$
      \item $x_i, y_i, r_i \leftarrow x_{i-2}-q_ix_{i-1}, y_{i-2}-q_iy_{i-1},
        r_{i-2}-q_ir_{i-1}$
      \item Ugorjunk (4)-re.
    \end{enumerate}
  \end{definition}

  \begin{exercise} Írj programot, ami a bővített Euklideszi algoritmust
    valósítja meg természetes számokra! Ellen\H orzéshez használható az 
    $\mathtt{xgcd}$ parancs.
  \end{exercise}
  
  \begin{definition}[(Lineáris) Diofantikus probléma]
    Az $a,b,c\in\mathbb{Z}$ számok esetén az $ax+by=c$ egyenletet az egész
    számok fölött (egész megoldásokat keresünk) lineáris Diofantikus 
    egyenletnek hívunk.
  \end{definition}

  A megoldások számának vizsgálatánál először észrevehet\H o, hogy $(a,b)$
  osztja a bal oldalt, hiszen $a$-nak és $b$=nek is osztója, így a jobb oldalt
  is kell osztania. Ez azt jelenti, hogy csak akkor van megoldás, ha $(a,b)|c$.
  Viszont ebben az esetben biztosan van megoldás hiszen a b\H ovített
  Euklideszi algoritmussal kaphatunk  egyet, ha annak kimenetét megszorozzuk
  $c/(a,b)$-vel ($x_0, y_0$). Ha van még további megoldás, akkor az felírható
  az $(x_0 + x', y_0 + y')$ alakban alkalmas $x',y'$ számokkal. Ekkor \[
    a(x_0+x')+b(y_0+y') = c = ax_0 + by_0,\] azaz \[ ax' = -by'. \]
  A jobb oldal osztható $b$-val, így a bal is, tehát \[
    \begin{array}{lclrl}
      \displaystyle b|ax' \Rightarrow \frac{b}{(a,b)}|x' & \Rightarrow &  
      x'= & &  \displaystyle t\frac{b}{(a,b)} \\
      \displaystyle ax' = -by' \Rightarrow at\frac{b}{(a,b)} = -by' & \Rightarrow &
      y'= & - & \displaystyle t\frac{a}{(a,b)}\\
    \end{array}\qquad (t\in\mathbb{Z}).\]
  Összefoglalva, ha van megoldás akkor végtelen sok van és egy tetsz\H oleges
  $(x_0, y_0)$ megoldásból a többit a 
  \[
    x_t = x_0 + t\frac{b}{(a,b)}\qquad y_t= y_0-t\frac{a}{(a,b)}\qquad(t\in\mathbb{Z})\]
  formulákkal kaphatjuk.

  \emph{Megjegyzés:} A lineáris Diofantikus probléma elképzelhet\H o egy úgy
  is, hogy az egyenlet egy egyenes egyenlete a síkon és a kérdés az, hogy
  ennek az egyenesnek van-e és mennyi metszéspontja van az egész számok segítségével
  készített ráccsal. A fenti megoldás itt annak felel meg, hogy megpróbáljuk 
  egész érték\H u eltolással elmozdítani az egyenes egy pontját az origóba. Ha
  sikerül az az jelenti, hogy a ráccsal közös pontok( az origón kívül) a
  meredekségnek $\frac{a}{b}=\frac{a/(a,b)}{b/(a,b)}$ megfelel\H o négyzetek 
  megfelel\H o csúcsai lesznek..

  \begin{exercise} Valósítsd meg a $\mathtt{LinDiofantianEq}$ osztályt a
    következőeknek megfelelően!
    \begin{itemize}
      \item Konstruktorában kell megadni az $a,b,c$ értékeket.
      \item Van egy $\mathtt{is\_solvable}$ függvénye.
      \item Fel tudja sorolni a megoldásokat egy $\mathtt{next\_solution}$ és
        egy $\mathtt{prev\_solution}$ függvény segítségével.
      \item Az első megoldás, amivel a $\mathtt{next\_solution}$ visszatér az
        legyen, amely esetén az $x$ a legkisebb nemnegatív szám.
      \item Csak egy megoldást tároljunk az objektum használata közben.
    \end{itemize}
  \end{exercise}

  \begin{exercise}
    Hányféleképpen tudunk kifizetni 100000 peng\H ot 47 és 79 peng\H os
    érmékkel?
  \end{exercise}


  \begin{exercise}
    Egy üzletben háromféle csokoládé kapható, 70, 130 és 150 forint
    egységárban. Hányféleképpen lehet pontosan 5000 forintért 50 
    darab csokoládét venni?
  \end{exercise}
    
  \begin{exercise}
    Írj programot, amely a három argumentuma ($a,b,c$) visszatér hány
    megoldása van az $ax+by=c$ diofantikus problémának a természetes 
    számok felett (nemnegatív megoldások)!
  \end{exercise}

  \begin{exercise}
    Írd meg a \[ \mathtt{multi(L, c, s=0)} \]
    függvényt, amelyre
    \begin{itemize}
      \item $\mathtt{L}$ lista elemei $a_0,a_1,\dots$;
      \item visszatérési érték a \[ \sum_{i=0}^{\mathtt{len(L)}} a_ix_i = c \]
        egyenlet nemnegatív egész megoldásainak száma $\mathtt{s=0}$ esetén,
        különben
      \item azon megoldások száma, amelyek még teljesítik a 
        \[ \sum_{i=0}^{\mathtt{len(L)}} x_i = s \]
        feltételt is.
    \end{itemize}
  \end{exercise}
\end{Section}

\begin{section}{Kongruenciák}
  %%% \begin{definition}[Kongruens]
    % Az $a$ és $b$ kongruens modulo $m$, ha 
  % \end{definition}
\end{section}

\end{Part*}

\begin{part}{Megoldások}

  \begin{section}{Számelmélet}

    \subsection{} Sok-sok megoldás elképzelhet\H o, például:
      \begin{sageexample}
          sage: def divides0(a,b):
          ....:     return (a/b).is_integer()
          sage: def divides1(a,b):
          ....:     return a % b == 0
          sage: def divides2(a,b):
          ....:     return (a//b)*b == a
          sage: def divides3(a,b):
          ....:     return (a/b).denom() == 1
          sage: def divides4(a,b): #there is room to improve
          ....:     if a == 0:
          ....:         return True
          ....:     b *= sign(b)
          ....:     if b == 1:
          ....:         return True
          ....:     q = b
          ....:     a *= sign(a)
          ....:     while q <= a:
          ....:         q <<= 1
          ....:     while a > b:
          ....:         q >>= 1
          ....:         a -= q  
          ....:         a *= sign(a)
          ....:     return a == 0 or a == b

          \end{sageexample}

      \end{section}

    \end{part}

    \end{document}
