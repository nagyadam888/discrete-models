\documentclass{amsbook}

% LANGUAGE: choose one of the following line and compile the doc
\usepackage[magyar]{babel}
%\usepackage[english]{babel}

\usepackage[languages={english,magyar}]{multilang}
\usepackage{multilang-sect}
\usepackage{lecture_notes_commands}

% set font encoding for PDFLaTeX, XeLaTeX, or LuaTeX
\usepackage{ifxetex,ifluatex}
\newif\ifxetexorluatex
\ifxetex
  \xetexorluatextrue
\else
  \ifluatex
    \xetexorluatextrue
  \else
    \xetexorluatexfalse
  \fi
\fi

\ifxetexorluatex
  \usepackage{fontspec}
\else
  \usepackage[T1]{fontenc}
  \usepackage[utf8]{inputenc}
  \usepackage{lmodern}
\fi
\usepackage{amsfonts}

\usepackage{hyperref}
\usepackage{sagetex}
% http://doc.sagemath.org/html/en/tutorial/sagetex.html

% For centering things
\usepackage{chngcntr}

\usepackage{bookmark}

\usepackage{version}
\newenvironment{version}{\relax}{\relax}
%\excludeversion{solution}



% The counting will restart for every part.
\counterwithin*{section}{part}

% There is no identation at the begining of paragragaphs
\setlength{\parindent}{0pt}

\Title{
  title/english=Applications of Discrete models,
  title/magyar =Diszkrét modellek alkalmazásai
}
\Author{
  author/english=Adam Nagy,
  author/magyar =Nagy Ádám
}

\begin{document}

\maketitle

\setcounter{part}{1}
\begin{Part*}{ title/english=Number Theory, title/magyar={Számelmélet}}
  
    \Wrap{
    content/english={
      The \emph{SageMath} application gives the whole structure of integer
      numbers as the \texttt{ZZ} object which behaves according to the "usual"
      mathematical definition.},
    content/magyar ={
      A \emph{SageMath} programcsomagban az egész számokat a \texttt{ZZ}
      objektummal kapjuk meg a ,,szokásos``  matematikai definíciónak
      megfelel\H en.}
  }
  \begin{sageexample}
    sage: type(ZZ)
  \end{sageexample}
  
  \Wrap{
    content/magyar={
      Természetesen nem kell minden esetben használnunk a konstruktort, ha egy
      egész számmal szeretnénk dolgozni, a rendszer automatikusan felismeri.
    },
    content/english={
      Fortunatelly there is no need to use the constructor when we want to
      create an integer, the system automaticly recognizes such numbers. 
    }
  }
  \begin{sageexample}
    sage: a,b = ZZ(4), 4
    sage: type(a) == type(b)
    sage: a == b
  \end{sageexample}

  \Wrap{
    content/magyar={Aritmetiaki m\H uveletek a ,,szokásosak``:},
    content/english={Arithmetic on integers:}
  }
  \begin{itemize}
    \item 
      \Wrap{
        content/magyar={Összeadás, kivonás},
        content/english={Addition, substraction}
      }: \texttt{+,-};
    \item 
      \Wrap{
        content/magyar={Szorzás, hatványozás},
        content/english={Multiplication, raising to a power}
      }: \texttt{*,\^};
    \item 
      \Wrap{
        content/magyar={Egész érték\H u osztás és maradékképzés},
        content/english={Division over the integers and residue}
      }: \texttt{//,\%}.
  \end{itemize}
  
  \Wrap{
    content/magyar={
      \emph{Megjegyzés:} A \texttt{/} m\H uvelet eredménye egy racionális
      szám, sőt valójában a jelenléte elég, hogy innentől racionálisként
      tekintsen a megadott adatokra.},
    content/english={
      \emph{Note:} The result of the \texttt{/} operand is a rational number
      and in general if this operand  is used then the result's type will 
      be rational.}
  }
  \begin{sageexample}
    sage: 2/3
    sage: type(2/3)
    sage: 1/1
    sage: type(1/1)
  \end{sageexample}

\begin{Section}{
  title/english=Divisor,
  title/magyar=Oszthatóság}

  \input{number-theory/01-divisor.tex}

\end{Section}

\begin{Section}{
  title/english=Congruent,
  title/magyar=Kongruencia}

  \begin{definition}[\Wrap{content/magyar=Kongruencia,content/english=Congruent}]
  \Wrap{
    content/magyar={Az $a$ és $b$ számok kongruensek modulo $m$ ($m>0$), azaz
      \[ a\equiv b \mod m\text{, amennyiben }m|(a-b).\]},
    content/english={The numbers of $a$ and $b$ are congruent modulo $m$ ($m>0$), i.e.
      \[a\equiv b(\mod m)\text{, if }m|(a-b).\]}
  }
\end{definition}

\Wrap{
  content/magyar={A kongruencia mint reláció reflexív, szimmetrikus és tranzitív is, azaz
    ekvivalenciareláció, így meghatározza az alaphalmaz egy osztályozását.},
  content/english={The congruence as a relation is reflexive, symmetric and transitive, thus
    an equivalence relation, which implies a partitioning of the set its defined over.}
}

\begin{exercise}
  \Wrap{
    content/magyar={Írj programot, amely egy egész számokat tartalmazó halmaz elemeit osztályozza
      modulo $m$, ahol az $m$ a második paraméter.},
    content/english={Write a program that sorting the elements of a given set (first argument)
      modulo $m$ (second argument).}
  }

  \begin{solution}
    \begin{sageexample}
      sage: def residue_sets(S,m):
      ....:     rs = {}
      ....:     for e in S:
      ....:         r = e % m
      ....:         if r in rs.keys():
      ....:             rs[r].add(e)
      ....:         else:
      ....:             rs[r] = set([e])
      ....:     return rs
    \end{sageexample}
  \end{solution}
\end{exercise}

\begin{definition}[\Wrap{content/magyar={Maradékrendszer}, content/english={Residue system}}]
  \Wrap{
    content/magyar={Egész számok esetén a kongruencia mint ekvivalenciareláció által
      meghatározott osztályokat \emph{maradékosztály}nak, míg rendszerüket \emph{maradékrendszer}nek
      nevezzük.},
    content/english={TODO}
  }
\end{definition}

\Wrap{
  content/magyar={Számolás során a maradékosztályokat egy-egy reprezentánsukkal szoktuk jelölni,
    például $m$ esetén gyakori a $0,1,\dots,m-1$ (legkisebb nem negatív reprezentások) vagy
    egész számok esetén a $-\lfloor \frac{m-1}{2}\rfloor,\dots,0,\dots,\lceil\frac{m-1}{2}\rceil$
    (legkisebb abszolút értékű reprezentások) használata.},
  content/english={TODO}
}

\begin{definition}[\Wrap{content/magyar={Redukált maradékrendszer}, content/english={Reduced residue system}}]
  \Wrap{
    content/magyar={Ha a maradékrendszerből elhagyjuk az összes olyan maradékosztályt melyek elemei
      nem relatív prímek a modulushoz, akkor megkapjuk a \emph{redukált maradékrendszer}t.},
    content/english={TODO}
  }
\end{definition}

\begin{definition}[\Wrap{content/magyar={Euler-féle $\varphi$ függvény}}]
  \Wrap{
    content/magyar={A $\varphi:\mathbb{N}\to\mathbb{N}$ függvényt az Euler-féle $\varphi$
      függvénynek nevezzűk, ha $\varphi(m)$ a modulo $m$ redukált maradékrendszerek száma, azaz},
    content/english={TODO}
  }
  \[\varphi(m) = \left|\{k\in\mathbb{Z}:1\le k< m\wedge (k,m)=1\}\right|.\]
\end{definition}

\Wrap{
  content/magyar={Ha $p$ egy prím és $n$ tetszőleges természetes szám, akkor a $\varphi(p^n) =p^n-p^{n-1}$
    könnyen kapható, hiszen pontosan minden $p$-edik maradékosztály tartalmaz $p$-vel osztható
    számokat, a többiben relatív prímek vannak $p$-hez és így $p^n$-hez is. Össztett számokkal való
    számoláshoz elég észrevenni, hogy a $\varphi$ számelméleti függvény multiplikatív, azaz
    relatív prím $a,b$ számokra $\varphi(ab)=\varphi(a)\varphi(b)$.},
  content/english={TODO}
}

\Wrap{
  content/magyar={A $\varphi(n)$ maximuma nyilvánvalóan $n-1$, viszont minimuma közel sem lineáris.},
  content/english={TODO}
}


\begin{sageexample}
  sage: P  = points([(k,euler_phi(k)) for k in range(1,1001)])
\end{sageexample}
\begin{figure}[h]
  \centering
  \sageplot[scale=.6]{P}
  \caption{Euler-féle $\varphi$ függvény értéke 1 és 1000 közötti számokra (\texttt{P}).}
\end{figure}

\begin{exercise}
  \Wrap{
    content/magyar={Írj programotfüggvényt, amely az Euler-féle $\varphi$ függvény értékét
      számolja ki! Ellenőrzéshez használható az \texttt{euler\_phi} parancs.},
    content/english={TODO}
  }

  \begin{solution}
    \begin{sageexample}
    # according to the definition
    sage: def ephi_01(m):
    ....:     p = 0
    ....:     for i in range(1,m):
    ....:         p += gcd(i,m) == 1
    ....:     return p

    #according to the previous note
    sage: def ephi_02(m):
    ....:     p = 1
    ....:     for (a,b) in factor(m):
    ....:         p *= a^(b-1)*(a-1)
    ....:     return p

    \end{sageexample}
  \end{solution}
\end{exercise}

\begin{definition}[\Wrap{content/magyar=Lineáris kongruenciák, content/english=Linear congruence}]
  \Wrap{
    content/magyar={Az $a,b$ egész és $m$ pozitív egész számok esetén az \[ax\equiv b\ (m)\] alakú
      kifejezéseket \emph{lineáris kongruenciának} hívjuk.},
    content/english={TODO}.
  }
\end{definition}

\Wrap{
  content/magyar={A kongruencia és oszthatóság definíciókat használva kapjuk, hogy alkalmas $y$-al},
  content/english={TODO}
}
\[ ax\equiv b\ (m) \Leftrightarrow m|ax-b \Leftrightarrow ax-b = my \Leftrightarrow ax-my=b.\]
\Wrap{
  content/magyar={Ez azt jelenti, hogy egy lineáris kongruencia megoldását megkaphatjuk a
    megfelelő lineáris diofantikus probléma megoldásával. Továbbá},
  content/english={TODO}
}
\begin{itemize}
  \item $(a,m)|b$
    \Wrap{
      content/magyar=szükséges és elégséges feltétel a megoldás létezésére;,
      content/english={TODO}
    }
  \item
    \Wrap{
      content/magyar={$acx\equiv bc\ (cm)$ kongruencia megoldásait megkaphatjuk az
        $ax\equiv b\ (m)$ kongruenca megoldásával;},
      content/english={TODO}
    }
  \item
    \Wrap{
      content/magyar={$(a,m)=1$ esetén mindkét oldalt oszthatjuk $(a,b)$-vel;},
      content/english={TODO}
    }
  \item
    \Wrap{
      content/magyar={$(a,m)=1$ és $(b,m)=c$ esetén a $ax\equiv b\ (m)$ kongruencia megoldásait
        kaphatjuk a $ax\equiv b/c\ (m/c)$ kongruencia megoldásával.},
      content/english={TODO}
    }
\end{itemize}

\begin{exercise}
  \Wrap{
    content/magyar={Írj eljárást lineáris kongruenciák megoldására! Ellenőrzéshez használható a
      \texttt{solve\_mod} parancs.},
    content/english={TODO}
  }
  \begin{solution}
    \begin{sageexample}
    sage: def lin_cong(a,b,m):
    ....:     (d,x,y) = xgcd(a,m)
    ....:     return x % m/d
    \end{sageexample}
  \end{solution}
\end{exercise}

\begin{definition}[\Wrap{content/magyar={Moduláris inverz}, content/english=Modular inverse}]
  \Wrap{
    content/magyar={Az $ax\equiv 1\ (m)$ kongruencia megoldását (ha van) az $a$ szám
      \emph{moduláris inverz}ének nevezzük modulo $m$.},
    content/english={TODO}
  }
\end{definition}

\begin{exercise}
  \Wrap{
    content/magyar={Írj programot, amely kiszámolja első paraméterének moduláris inverzét modulo
      a második paraméter! Ellenőrzéshez használható az \texttt{inverse\_mod} parancs},
    content/english={TODO}
  }

  \begin{solution}
    \begin{sageexample}
    sage: def modinv(a,m):
    ....:     (d,x,y) = xgcd(a,m)
    ....:     if d == 1:
    ....:         return x % m
    ....:     else:
    ....:         return None
    \end{sageexample}
  \end{solution}
\end{exercise}

\begin{definition}[\Wrap{content/magyar=Lineáris kongruencia-rendszer, content/english=System of congruences}]
  \Wrap {
    content/magyar={Legyen $1<n\in\mathbb{N}$, $a_i,b_i\in\mathbb{Z}$ és $1<m_i\in\mathbb{N}$
      ($1\le i\le n$). Ekkor a \[a_ix\equiv b_i\ (m_i) \quad(1\le i\le n)\] kongruenciák összeségét
      \emph{lineáris kongruencia-rendszer}nek hívunk és csak olyan $x$ egész számot tekintünk
      megoldásnak, amely mindegyiknek külön-külön is megoldása.},
    content/english={TODO}
  }
\end{definition}

\Wrap{
  content/magyar={ A kongruenciarendszerek megoldásának megkereséséhez tekintsünk csak két
    kongruenciát és első lépésként oldjuk meg őket külön-külön. Ezek után a feladat az
    \[ x\equiv c_1 (m_1)\text{ és } x\equiv c_2 (m_2), \] kongruenciarendszer megoldásainak
    megtalálása. A fentieknek megfelelően ez azt jelenti, hogy arra alkalmas $y_1$ és $y_2$
    számokkal \[
      \left.
      \begin{array}{ccc}
        m_1|x-c_1 & \Leftrightarrow & x = c_1 + m_1y_1 \\
        m_2|x-c_2 & \Leftrightarrow & x = c_2 + m_2y_2
      \end{array}
      \right\}\Rightarrow c_1-c_2 = m_1y_1 - m_2y_2,
    \]
    ami tetszőleges $c_1, c_2$ esetén csak akkor lehetséges, ha $m_1$ és $m_2$ relatív prímek.
  },
  content/english={TODO}
}

\Wrap{
  content/magyar={Az általános megoldás megtalálásához az előzőek alapján tegyük fel, hogy
    $(m_1, m_2)=1$ és keressük $x$-et $x=x_1+x_2$ alakban, ahol \[
      \begin{array}{ccccccccc}
        x_1 & \equiv & c_1 & (m_1) & \quad & x_1 & \equiv & 0   & (m_2) \\
        x_2 & \equiv & 0   & (m_1) & \quad & x_2 & \equiv & c_2 & (m_2).
      \end{array}
    \]
    Ebből $x_1$-re $m_1|x_1-c_1$ és $m_2|x_1$, azaz $m_1u_1 = x_1-c_1$ és $m_2v_2 = x_1$, tehát
    ha $m_1u+m_2v=1$, akkor \[
      c_1 = m_1u_1-m_2v_1 = m_1uc_1 + m_2vc_1.
    \]Így $x_1=c_1-m_1uc_1 = m_2vc_1$ és hasonlóan $x_2=c_2-m_2vc_2=m_1uc_2$. Ez alapján azt
    kaptuk, hogy a fenti két kongruenciából álló rendszer egy megoldása \[ x = c_1m_2v + c_2m_1u. \]
  },
  content/english={TODO}
}

\Wrap{
  content/magyar={Az nyilvánvaló, hogy az $x+km_1m_2$ is megoldás lesz tetszőleges $k$ egész
    számra, továbbá a megoldás egyértelmű is modulo $m_1m_2$, mivel bármely két megoldás
    különbsége 0 modulo $m_1$ és $m_2$ is, azaz a megoldások közötti különbség a $[m_1,m_2]$
    többszöröse kell hogy legyen. },
  content/english={TODO}
}

\begin{theorem}[\Wrap{content/magyar=Kínai maradéktétel (KMT), content/english=Chinese remainder theorem (CRT)}]
  \Wrap{
    content/magyar={Legyenek $m_1,m_2,\dots,m_n$ egynél nagyobb páronként relatív prím természetes
      számok. Ekkor az $x\equiv c_i\ (m_i)$ $(1\le i\le n)$ kongruenciarendszernek van megoldása
      és a megoldások kongruensek modulo $m_1m_2\dots m_n$, bármely egész $c_1, c_2,\dots c_n$ egész
      esetén.},
    content/english={TODO}
  }
\end{theorem}

\begin{exercise}
  \Wrap{
    content/magyar={Írj eljárást, amely a kínai maradéktétel megoldását állítja elő. Az
      programnak két lista típusú bemenete legyen, az egyik a kínai maradéktételnél
      szereplő $c$ számok a másik pedig a (páronként relatív prím) modulusok. Ellenőrzéshez
      használható a \texttt{crt} parancs.},
    content/english={TODO}
  }

  \begin{solution}
    \begin{sageexample}
    sage: def chinese(C,M):
    ....:     if len(C) != len(M):
    ....:         return None
    ....:     c, m = C[0], M[0]
    ....:     for i in range(1, len(C)):
    ....:         (g, u, v) = xgcd(m, M[i])
    ....:         if g != 1:
    ....:             return None
    ....:         c = (c*M[i]*v + C[i]*m*u)
    ....:         m *= M[i]
    ....:         c %= m
    ....:     return c
    \end{sageexample}
  \end{solution}
\end{exercise}

\begin{exercise}
  \Wrap{
    content/magyar={Írj eljárást amely lineáris kongruencia-rendszereket old meg! A programnak
      három lista típusu bemenete van: a bal oldalak együtthatóinak, a jobb oldalaknak és a
      modulusoknak listái.},
    content/english={TODO}
  }

  \begin{solution}
    \begin{sageexample}
    sage: def lin_cong_sys(A,B,M):
    ....:     if len(A) != len(B) or len(B) != len(C):
    ....:         return None # should raise some exception
    ....:     c, m = 0, 1
    ....:     for  i in range(len(A)):
    ....:         #sole the ith equation
    ....:         (d, x, _) = xgcd(A[i], M[i])
    ....:         if B[i] % d != 0:
    ....:             return None
    ....:         x = (x*B[i]/d) % M[i]
    ....:         #add to the solution
    ....:         (g, u, v) = xgcd(m, M[i])
    ....:         if (c-x) % g != 0:
    ....:             return None
    ....:         c = (c*M[i]*v + x*m*u)/g
    ....:         m *= M[i]
    ....:         c %= m
    ....:     return c
    \end{sageexample}
  \end{solution}
\end{exercise}

\Wrap{
  content/magyar={A lineáris egyenletek mellett természetesen magasabb rendű és más típusú
    egyenletek is elképzelhetőek. Ezek megoldása általában más problémákat vet fel mint valós
    vagy akár komplex megfelelőjük, de mivel a keresett megoldás egy véges halmazban van, a
    legrosszabb esetben is megkaphatjuk a megoldást végigpróbálva az összes lehetséges értéket.},
  content/english={TODO}
}

\begin{exercise}[\Wrap{content/magyar=Kvardratikus maradékok, content/english=Quadratic residues}]
  \Wrap{
    content/magyar={Írj programot, amely egy adott $m$ esetén megadja azon $b$ $0$ és $m-1$
    közötti számok halmazát, amelyek esetén az \[ x^2 \equiv b\ (m) \] egyenlet megoldható.},
    content/english={TODO}
  }
  \begin{solution}
    \begin{sageexample}
    sage: def quadratic_residues(m):
    ....:     qrs = set()
    ....:     for i in range(m):
    ....:         qrs.add(i^2 % m)
    ....:     return qrs
    \end{sageexample}
  \end{solution}
\end{exercise}

\Wrap{
  content/magyar={Matematikában és az informatikai alkalmazások területén is fontos szerepe van a
    \[ a^x\equiv b\ (m) \] típusú (logaritmushoz hasonló) egyenleteknek. },
  content/english={TODO}
}

\begin{theorem}[(Kis) Fermat-tétel]
  Ha $p$ prím és $a$ tetszőleges egész szám, akkor\[ a^{p-1}\equiv 1\ (p).\]
\end{theorem}

\begin{theorem}[Euler-Fermat-tétel]
  Ha $a$ és $m$ relatív prímek, akkor \[ a^{\varphi(m)}\equiv 1\ (m), \] ahol $\varphi$ az
  Euler-féle $\varphi$ függvény.
\end{theorem}

\begin{definition}[RSA asszimetrikus titkosítás]
  Általában egy asszimetrikus titkosítási sémánál két kulcs áll rendelkezésre (egy publikus és egy
  privát) és a két kucsot egymás után használva visszakapjuk az üzenetet. \emph{RSA} séma esetén
  \begin{itemize}
    \item választunk két elég nagy és megfelelő formájú $p$, $q$ prímet,
    \item egy $e>1$ kitevőt és
    \item számoljuk ki $n=pq$-t, illetve
    \item egy $d$ egész számot, melyre $ed\equiv 1\ (\varphi(n)=(p-1)(q-1))$.
  \end{itemize}
  A publikus kulcs $(n,e)$, a privát kulcs $(n,d)$ lesz és egy $m<n$ szám mint üzenet titkosított
  formáját kapjuk az $s = m^d \mod n$ kiszámolásával. A visszafejtés az Euler-Fermat-tétel
  használatával \[ s^d\equiv (m^d)^e\equiv m^{ed}\equiv m^{\varphi(m)q+1}\equiv m\ (n). \]
\end{definition}

\begin{exercise}
  \Wrap{
    content/magyar={Írj osztályt, amely addott publikus paraméterek esetén megvalósítja a RSA sémát!},
    content/english={TODO}
  }

  \begin{solution}
    \begin{sageexample}
    sage: class RSA(object):
    ....:     #this is just a dummy implementation, not for actual use
    ....:     def __init__(self, length):
    ....:         # uniformly choosen prime is not a good idea in real life
    ....:         p = random_prime(2^(length-2), lbound=2^(length-3))
    ....:         q = random_prime(2^(length+2), lbound=2^(length+1))
    ....:         self.__n = p*q
    ....:         self.__phin = (p-1)*(q-1)
    ....:         self.__e = 3 #should choose this more carefully
    ....:         while gcd(self.__e, self.__phin) != 1:
    ....:             self.__e += 2
    ....:         self.__d = inverse_mod(self.__e, self.__phin)
    ....:     def public_key(self):
    ....:         return (self.__n, self.__e)
    ....:     @staticmethod
    ....:     def encrypt(pubkey, message):
    ....:         return power_mod(message, pubkey[1], pubkey[0])
    ....:     def decrypt(self, secret):
    ....:         return power_mod(secret, self.__d, self.__n)
    \end{sageexample}
  \end{solution}
\end{exercise}

\begin{definition}[Diszkrét logarimus probléma]
  Vegyünk egy $p$ prímet és egy olyan $g$ számot, amely hatványaival modulo $p$ előállítja az
  összes $p$-nél kisebb pozitív számot. Ekkor egy $a$ esetén a $g^a\mod p$ értékből $a$
  meghatározását diszkrét logaritmus problémának hívjuk.
\end{definition}

\begin{definition}[Diffie-Hellman kulcscsere]
  A diszkrét logaritmus problémánál használt $p$ és $g$ publikus paramétereket használva két
  kommunikációs fél (Alice és Bob) tud közös értékben (kulcs) megállapodni Diffie-Hellman
  sémát használva. A séma során mindkét fél választ egy-egy véletlen értéket (titok) és számolják a
  $g^a$ és $g^b$ publikus értékeket. Ezek alapján mindketten ki tudják számolni a közös kulcsot:
  \[g^{ab} = (g^a)^b = (g^b)^a.\]
\end{definition}

\begin{exercise}
  \Wrap{
    content/magyar={Írj programot, amely egy Diffie-Hellman kulcscsere folyamatát szemlélteti.},
    content/english={TODO}
  }

  \begin{solution}
    \begin{sageexample}
    sage: class DH_participant(object):
    ....:     def __init__(self, p, g):
    ....:         self.__p = p
    ....:         self.__g = g
    ....:         self.__x = randint(1, p-1)
    ....:     def get_pub(self):
    ....:         return power_mod(self.__g, self.__x, self.__p)
    ....:     def calculate_common_key(self, pub_of_other):
    ....:         return power_mod(pub_of_other, self.__x, self.__p)
    ....: Alice = DH_participant(65537, 2)
    ....: Bob   = DH_participant(65537, 2)
    ....: #common key
    ....: Alice.calculate_common_key(Bob.get_pub())
    ....: Bob.calculate_common_key(Alice.get_pub())
    \end{sageexample}
  \end{solution}
\end{exercise}

% TODO put the cryptograhpy part in the coding theory as separate section

\end{Section}


\end{Part*}

\setcounter{part}{2}
\setcounter{section}{0}
\begin{Part*}{ title/english=Coding Theory, title/magyar=Kódoláselmélet}
    % TODO write introduction to coding theory

\Wrap{
  content/english={TODO},
  content/magyar ={
    A kódoláselmélet kódok tulajdonságait vizsgálja abból a szempontból, hogy mennyire
    felelnek meg a különféle alkalmazási területeken. Ezen vizsgálat során felmerülő feladatok
    általánosíthatók a következő modellel: egy küldő egy vagy több üzenetet probál meg eljuttatni
    egy fogadóhoz valamilyen tulajdonságokkal rendelkező csatornán keresztül.}
}

\begin{Section}{
  title/english=Source coding,
  title/magyar=Forráskódolás}

  \Wrap{
  content/magyar={
    A forráskódolás a kódok hosszával foglalkozik, vagyis azzal a kérdéssel, hogy az adott
    mennyiségű információt mekkora mennyiségű adattal tudjuk tárolni. Ehhez szükségünk van az
    információ alapegységére $r$, amely bináris esetben 2.
  },
  content/english={TODO}
}

\begin{definition}[\Wrap{content/magyar=Entrópia, content/english=Entropy}]
  \Wrap{
    content/magyar={
      A kódolt és kódolatlan üzenetek ($m$) karakterekre bonthatóak, melyek önmagukban is, de
      leginkább az üzenetben elfoglalt helyük segítségével információt tárolnak. Arra hogy mennyi
      információt hordoz egy üzenet a hosszához viszonyítva a karakterek rendezetlensége utal.
      Például egy egyetlen karaktert ismételgető forrás által küldött üzenet információmennyisége
      kisebb mint egy olyané ami ugyanolyan hosszú, több karaktert használ és a karaktereket
      valamilyen bonyolutabb szabály szerint fűzi egymás után. Erre a rendezetlenségre és így a
      relatív információmennyiségre utal az \emph{entrópia}, amely ha az egyes karakterek
      előfordulásának valószínűsége $p_1,p_2,\dots,p_n$ $m$-ben, akkor értéke
      \[ H_r(m) = -\sum_{i=1}^n p_i\log_r p_i. \]},
    content/english={TODO}
  }
\end{definition}

\Wrap{
  content/magyar={Az entrópia értéke akkor a legkisebb ($0$), ha az üzenet csak egy karaktert
    tartalmaz; és akkor a legnagyobb ($\log_r n$), ha minden üzenet azonos valószínüséggel szerepel.
    Ebből közvetlenül tudunk következtetni, hogy egy szöveg mennyire ,,tömör'', mivel az
    entrópia megadja hogy a karaktereket mennyire ,,jól'' alkalmazzuk adott hosszon.
 },
  content/english={TODO}
}

\Wrap{
  content/magyar={A kódolás során legfontosabb szempont a dekódolhatóság, azaz egy kódhalmaz
    (kódszavakat tartalmazó halmaz) azon tulajdonsága, hogy bármely belőle készített kódolt üzenet
    egyértelműen dekódolható-e. Azonban ennek ellenőrzése nem minden esetben könnyű, így szokás a
    kódhalmazra (kódra) vonatkozó következő fogalmakat definiálni.},
  content/english={TODO}
}

\begin{definition}[\Wrap{content/magyar={Felbontható, egyenletes, vesszős és prefix kód}, content/english={TODO}}]
  \Wrap{
    content/magyar={Legyen a kódszavak ábécéje $B$ és $\alpha, \beta, \gamma\in B^*$ az ábécé
      feletti szavak (nem feltétlenül kódszavak). A kód ekkor},
    content/english={TODO}
  }
  \begin{itemize}
    \item\Wrap{
      content/magyar={\emph{felbontható}, ha bármely szöveg egyértelműen dekódolható;},
      content/english={TODO}}
    \item\Wrap{
      content/magyar={\emph{egyenletes}, ha minden kódszó azonos számú karaktert tartalmaz;},
      content/english={TODO}
    }
    \item\Wrap{
    content/magyar={\emph{vesszős}, ha minden kódszó felírható az $\alpha\gamma$ alakban és
      ha $\alpha\gamma\beta$ kódszó, akkor a $\beta = \varepsilon$, ahol $\varepsilon$ az
      egyeteln karaktert sem tartalmazó szó és $\gamma\neq\varepsilon$;},
    content/english={TODO}
    }
    \item\Wrap{
    content/magyar={\emph{prefix}, ha a kódszavak halmaza prefixmentes, azaz ha az
      $\alpha\neq\varepsilon$ és $\alpha\beta$ is kódszó, akkor $\beta=\varepsilon$;},
    content/english={TODO}
    }
  \end{itemize}

\end{definition}

\begin{definition}[\Wrap{content/magyar=Betűnkénti kódolás, content/english={TODO}}]
  A kódolás betűnként történik, ha a szöveg $A$ ábécéje és a kódszavak $B$ halmaza között létezik
  egy $\varphi\in A\to B$ injektív (minden értéket felvesz pontosan egyszer) leképezés.
\end{definition}

% TODO remove this when there is other type of source coding introduced in the document
\Wrap{
  content/magyar={A továbbiakban csak betűnkénti kódolásról fogunk beszélni.},
  content/english={TODO}
}

\begin{definition}[\Wrap{content/magyar=Kódfa, content/english=Codetree}]
  \Wrap{
    content/magyar={Egy kódhalmaz esetén a kódszavak felírhatóak egy fa segítségevel. A fa csúcsai
      szavak (nem feltétlenül kódszavak), az éleit pedig a kódszvak lehetséges karaktereivel
      címkézzük. A fa
      gyökerében az üres szó szerepel és egy szóhoz tartozó csúcs leszármazottai azok a szavak,
      amelyeket úgy kapunk, hogy a szó után írjuk az élen szereplő karaktert. A kódfa az a
      legkevesebb csúcsot tartalmazó ilyen tulajdonságú fa, ami tartalmazza az összes kódszót.}
  }
\end{definition}

\begin{sageexample}
  sage: def make_code_tree(C):
  ....:     G = DiGraph()
  ....:     for c in C:
  ....:         prev = ''
  ....:         for i in range(1,len(c)+1):
  ....:             G.add_edge(prev, c[0:i], c[i-1])
  ....:             prev = c[0:i]
  ....:     return G
  sage: C = {'1011', '1100', '0110', '1110', '1010', '0101', '101'}
  sage: G = make_code_tree(C)
  sage: d = {'#00FF00': [v for v in G.vertices() if v not in C],
  ....:      '#FF0000': list(C)}
  ....: GP = G.plot(layout='tree', vertex_size=2000,
  ....:             vertex_color=d, edge_labels=True)
\end{sageexample}

\begin{figure}[ht]
  \centering
  \sageplot[scale=.4][png]{GP, figsize=10}
  \caption{Kódfa a \texttt{C} kód esetén (\texttt{GP.show(figsize=10)}), kódszvak pirossal jelölve}
\end{figure}

\begin{theorem}[McMillan]
  \Wrap{
    content/magyar={Ha egy felbontható kód kódszavainak hossza rendre $\ell_1, \ell_2,\dots,
    \ell_n$ és a kódszavak ábécéjének elemszáma $r$, akkor \[\sum_{i=1}^n r^{-\ell_i}\le 1.\]},
    content/english={TODO}
  }
\end{theorem}

\Wrap{
  content/magyar={A tétel alapján a felbontható kód szavainak hosszára kapunk alsó korlátot, azaz
    arra, hogy adott paraméterek mellett legalább milyen hosszúaknak kell lennie egy felbontható
    kód szavainak.},
  content/english={TODO}
}

\begin{theorem}[Kraft]
  \Wrap{
    content/magyar={A \emph{McMillan} tétel megfordítása is igaz, sőt ha az $\ell_1,\ell_2,\dots,
      \ell_n$ olyan számok, amelyek megfelelnek a McMillan tételnél adott egyenlőtelnségnek, akkor
      konstruálható olyan prefix (így felbontható) kód, amelynek szóhosszai az adott számok.},
    content/english={TODO}
  }
\end{theorem}

\Wrap{
  content/magyar={A tétel következménye, hogy a felbontható de nem prefix kódok jelentősége nem
    nagy, hiszen az ilyen kódok helyett adható egy ugyanolyan tulajdonságokkal rendelkező prefix
    kód is.},
  content/english={TODO}
}

\begin{theorem}[\Wrap{content/magyar=Shannon zajmentes csatornára, content/english=Shannon's source conding}]
  \Wrap{
    content/magyar={ Legyen egy kód átlagos szóhosszúsága $\overline{\ell}$, az egyes szavak
      relatív gyakorisága $p_1,p_2,\dots,p_n$. Ekkor \[ H_r(p_1,p_2,\dots,p_n)\le\overline{\ell} .\]},
    content/english={TODO}
  }
\end{theorem}

\begin{definition}[\Wrap{content/magyar=Optimális kód, content/english=Optimal code}]
  \Wrap{
    content/magyar={Egy (betűnkéti) kódot optimálisnak nevezünk, ha az előző tétel jelöléseivel
      \[ \overline{\ell}\le H_r(p_1,p_2,\dots,p_n)+1.\]}
  }
\end{definition}

\Wrap{
  content/magyar={Egy optimális kód a fentieknek megfelelően a gyakori karakterekhez rövid
  kódszavakat, míg a ritkákhoz rövidebbet rendel; így a kódolt szöveg információmennyisége relatív
  nagy lesz. Érdemes azonban megjegyezn, hogy nem betűnkénti kódolás esetén nagyobb mértékű
  tömörítés (rövidebb reprezentáció)is elérhető.}
}

\Wrap{
  content/magyar={A továbbiakban három konstrukciót fogunk mutatni optimális kód
    előállítására, melyek közül igazából csak a Huffman-féle kódkonstrukció garantálja az optimális
    kódot. Az első kettő esetén csak azt tudjuk garantálni, hogy a kapott kód közel van az
    optimálishoz, így ezeket csak szuboptimális kódnak is nevezhetjük.},
  content/english={TODO}
}

\Wrap{
  content/magyar={Mindhárom konstrukciónál feltesszúk, hogy adott egy szöveghez tartozó egyes
    karakterekre számolt $p_1\ge p_2\ge\dots,\ge p_n$ relatív gyakoriságok.}
}
\begin{definition}[\Wrap{content/magyar=Shannon-kód, content/english=Shannon-code}]
  A szöveghez tartozó \emph{Shannon-kód}ot a kvetkező konstrukcióval kapjuk:
  \begin{enumerate}
    \setcounter{enumi}{-1}
    \item Rendezzük a gyakoriságokat csökkenő sorrendbe.
    \item Számoljuk ki a leendő kódhosszokat $\ell_i=\lceil -\log_r p_i\rceil$ ($1\le i\le n$).
    \item Az $i$-edik karakterhez tartozó kódszót megkapuk a
      \[\left\lfloor 2^{\ell_i}\sum_{j=0}^{i-1}p_i\right\rfloor \] szám $\ell_i$ hosszú bináris
      reprezentációjaként.
  \end{enumerate}
\end{definition}

\begin{definition}[\Wrap{content/magyar=Fano-kód, content/english=Fano-code}]
  A kód előállítása során egy kódfát fogunk építeni felülről lefelé.
  \begin{enumerate}
    \setcounter{enumi}{-1}
    \item Rendezzük a gyakoriságokat csökkenő sorrendbe.
    \item Osszuk fel két részre a kapott valószínűség sorozatokat úgy, hogy a bal oldal
      elemeinek együttes valószínűsége a lehető legközelebb legyen a jobb oldal valószínűségeinek
      összgéhez.
    \item A bal oldalon szereplő valószínűségekhez tartozó kódszók $0$-val, a jobb oldalhoz
      tartozók $1$-gyel kezdődnek/folytatódnak.
    \item Végezzük el az előző két pontban ismertetett eljárást rekurzívan mindkét részsorozatra,
      amíg 1 hosszú sorozatokat nem kapunk.
  \end{enumerate}
\end{definition}

\begin{definition}[\Wrap{content/magyar=Huffman-kód, content/english=Huffman-code}]
  A kód előállítása során itt is egy kódfát fogunk építeni, de most alúlról felfelé.
  \begin{enumerate}
    \item Rendezzük a gyakoriságokat csökkenő sorrendbe.
    \item Vonjuk össze a két legkisebb gyakoriságot és a hozzájuk tartozó karakterek közül a
      kisebb gyakorisággal rendelkező $0$-val, a másik $1$-el végződik.
    \item Az összevonás eredményét helyezzük el a sorozatba és ismételjük meg a teljes eljárást
      rekurzívan addig, amíg már csak egy gyakoriság lesz.
  \end{enumerate}
\end{definition}

\Wrap{
  content/magyar={Tekintsük példaként azt a szöveget, ami az $A,B,C,D$ és $E$ karaktereket
    tartalmazza rendre $16,8,7,6,3$ számossággal.},
  content/english={TODO}
}
\begin{sageexample}
  sage: abc = ['A', 'B', 'C', 'D', 'E']
  ....: num = [ 16,   8,   7,   6,   3]
  ....: s = sum(num)
  ....: P = [(k/s).n(digits=3) for k in num]
\end{sageexample}

\begin{table}[ht]
  \centering
  \sage{table([abc, num, P], header_row=True, header_column=["ch", "db", "$p_i$"], frame=True)}
\end{table}

\Wrap{
  content/magyar={Shannon-kód számolása:},
  content/english={TODO}
}
\begin{sageexample}
  sage: ell = [ceil(-log(p, 2)) for p in P]
  ....: sc = [(floor(sum(P[0:i]) << ell[i])).binary().rjust(ell[i],'0')
  ....:       for i in range(len(abc))]
\end{sageexample}

\begin{table}[ht]
  \centering
  \sage{table([abc, P, ell, sc], header_row=True, header_column=["ch", "$p_i$", "$\ell_i$",
  "code"], frame=True)}
\end{table}

\Wrap{
  content/magyar={Fano-kód számolása (szemléltetéssel ami bonyolultabb mint maga az algoritmus)},
  content/english={TODO}
}

\begin{sageexample}
  sage: G = DiGraph()
  ....: maxdepth = 4
  ....: fc = ['' for a in abc]
  ....: def split(u, v, parent, depth, code):
  ....:     if maxdepth <= depth:
  ....:         return
  ....:     if u >= v:
  ....:         fc[u] = code;
  ....:         return
  ....:     i, j = u, v
  ....:     rp, lp = P[i], P[j]
  ....:     rc, lc = abc[i], abc[j]
  ....:     while i+1 < j:
  ....:         if rp < lp:
  ....:             i += 1
  ....:             rp, rc = rp+P[i], rc+abc[i]
  ....:         else:
  ....:             j -= 1
  ....:             lp, lc = lp+P[j], lc+abc[j]
  ....:     rc, lc = rc + '\n' + str(rp), lc + '\n' + str(lp)
  ....:     split(u, i, rc, depth+1, code+'0')
  ....:     G.add_edge(parent, rc, '0')
  ....:     split(j, v, lc, depth+1, code+'1')
  ....:     G.add_edge(parent, lc, '1')
\end{sageexample}


\Wrap{
  content/magyar={1. lépés után:},
  content/english={TODO}
}

\begin{sageexample}
  sage: maxdepth=1
  ....: G = DiGraph()
  ....: split(0, len(abc)-1, '', 0, '')
  ....: GP = G.plot(layout='tree', edge_labels=True,
  ....:             vertex_size=4000, figsize=3, vertex_color='white')
\end{sageexample}

\begin{figure}[ht]
  \centering
  \sageplot[scale=.5][png]{GP}
  \caption{Fano-kód előállításának 1. iterációja (\texttt{GP})}
\end{figure}

\Wrap{
content/magyar={2. lépés után:},
content/english={TODO}
}

\begin{sageexample}
  sage: maxdepth=2
  ....: G = DiGraph()
  ....: split(0, len(abc)-1, '', 0, '')
  ....: GP = G.plot(layout='tree', edge_labels=True,
  ....:             vertex_size=4000, figsize=5, vertex_color='white')
\end{sageexample}

\begin{figure}[ht]
  \centering
  \sageplot[scale=.5][png]{GP}
  \caption{Fano-kód előállításának 2. iterációja (\texttt{GP})}
\end{figure}
\Wrap{
content/magyar={3. lépés után:},
content/english={TODO}
}

\begin{sageexample}
  sage: maxdepth=4
  ....: G = DiGraph()
  ....: split(0, len(abc)-1, '', 0, '')
  ....: GP = G.plot(layout='tree', edge_labels=True,
  ....:             vertex_size=4000, figsize=7, vertex_color='white')
\end{sageexample}

\begin{figure}[ht]
  \centering
  \sageplot[scale=.5][png]{GP}
  \caption{Fano-kód előállításának 3. iterációja (\texttt{GP})}
\end{figure}

\begin{table}[ht]
  \centering
  \sage{table([abc, P, fc], header_row=True, header_column=["ch", "$p_i$", "code"], frame=True)}
\end{table}

\Wrap{
  content/magyar={Huffman-kód esetén:},
  content/english={TODO}
}

Házi feladat

%TODO with a better example that not too big and gives different result

\end{Section}

\begin{Section}{
  title/english=Channel coding,
  title/magyar=Hibajelző és hibajavító kódolás}

  \Wrap{
  content/magyar={A kódolási feladatok esetén a kódolt üzenetnek egy csatornán keresztül kell
    eljutnia a fogadóhoz. Ez a csatorna lehet zajmentes, azaz garantált az, hogy amit a küldő a
    csatornába juttatott a fogadó hiba nélkül megkapja. Sajnos a valós alkalmazások esetén nincs így,
    alacsony kommunikációs szinten nem tudjuk vagy nem éri meg garantálni a bithelyes áramlást.
    A megoldás az, hogy olyan ún.~hibakorlátozókódot kunstruálunk, ami képes jelezni és/vagy
    javítani az átvitel közben keletkezett hibát. Az ilyen kódokok konstruálása általában nem
    egyszerű feladat és több tényezőt is figyelembe kell venni, mint például
    \begin{itemize}
        \item hijelző és hibajavító képesség;
        \item mennyivel lesz hosszabb a kódolt adat;
        \item mennyibe ,,kerül'' a kódolás és/vagy a dekódolás;
        \item milyen típusú (pl. csomókban vagy elszórtan) és eloszlású hibára számíthatunk.
    \end{itemize}
    },
  content/english={TODO}
}

\begin{definition}[\Wrap{content/magyar={(Pontosan) $t$-hibajelző kód}, content/english=TODO}]
  \Wrap{
    content/magyar={Egy kódot \emph{$t$-hibajelző}nek nevezünk, ha bármely $t$ hibát képes
      jelezni és \emph{pontosan $t$-hibajelző}, ha legfeljebb $t$ hibát tud biztosan észlelni,
      azaz van olyan $t+1$ hiba, amit már nem. },
    content/english={TODO}
  }
\end{definition}

\begin{definition}[\Wrap{content/magyar={(Pontosan) $t$-hibajavító kód}, content/english=TODO}]
  \Wrap{
    content/magyar={Egy kódot \emph{$t$-hibajavító}nek nevezünk, ha bármely $t$ hibát képes
      javítani és \emph{pontosan $t$-hibajavító}, ha legfeljebb $t$ hibát tud biztosan javítani,
      azaz van olyan $t+1$ hiba, amit már nem. },
    content/english={TODO}
  }
\end{definition}

\begin{definition}[\Wrap{content/magyar={Ismétléses-kód}, content/english={TODO}}]
    Talán a legegyszerűbb kódkonstrukció közé tartozsik az \emph{ismétléses-kód}, ami esetén
    minen egyes karaktert $1<k$-szor megismétlünk. Például a $01001$-ből $000111000000111$ lesz, ha
    $k=3$.
\end{definition}

Látható az ismétléses kód pontosan $k-1$ hibát képes jelezni, hiszen ha a $k$ hiba egyetlen
kódolás előtti karakterhez tartozik, akkor azt hibásan fogja dekódolni. A konstrukció hibája
is egyértelmű, $t$ hiba jelzéséhez $t+1$-szer hosszabb kódot készít. t=1 esetén a duplázó kóddal
megegyező hibajelző képeséggel (a definíció szerint) a paritásbites kódolással.

\begin{definition}[Paritásbites kód]
    A paritásbites kódot úgy kapjuk, hogy minden bináris szót kiegészítünk egy bittel annak
    megfelelően, hogy a benne lévő 1-esek száma páros vagy páratlan. Ha a páratlan számú egyesek
    esetén 1-et írunk a szóhoz különben 0-t, akkor \emph{párosra kiegészített paritásbites
    kódolás}t kajuk, míg ha pont fordítva járunk el akkor a \emph{páratlanra} egészítünk ki.
\end{definition}

A paritásbittel való kiegészítés 1 hibát tud észlelni, mivel már két bit változása esetén ismét
érvényes kódszót kapunk.

Észrevehető, hogy a hibajelző képesség attól függ, hogy legalább hány változtatás szükséges ahhoz,
hogy érvényes kódszót kapjunk. Ehhez a kapcsolat pontos kimondásához ad segítséget a következő
definíció.

\begin{definition}[Hamming távolság]
    Egy kód két szava közötti \emph{Hamming távolság}án $d(u,v)$ azon poziciók számát értjük, ahol a
    két szó eltér egymástól. Például $d(0110,1011) = 3$. A teljes kód távolságán a kódszavak
    távolságának minimumát értjük értjük, azaz \[ d(C)= \min\{d(u,v)|u,v\in C\}.\]
\end{definition}

Ha egy kód távolsága $d$, akkor az pontosan $d-1$ hibajavító, hiszen $d-1$ módosítás esetén biztosan
nem kaphatunk érvényes kódszót, de van olyan szópár, amely $d$ módosítással felcserélhetők.

A hibajavító képességhez először meg kell állapodni abban, hogy hogyan szeretnénk javítani a
hibákat, azaz meg kell állapodni abban a leképezésben, ami a nem kódszavakat kódszavakra képezi le.
Ehhez is a távlság fogalmát fogjuk használni.

\begin{definition}[Minimális súlyú dekódolás]
    A \emph{minimális súlyú dekódolás} esetén a fogadott nem kódszavakat úgy dekódoljuk, mintha a
    hozzá legközlebbi érvényes kódszót kaptuk volna, ha van ilyen.
\end{definition}

Minimális súlyú dekódolás esetén a hibajavító képesség már kónnyen megadható a kód távolságának
ismeretében. Ha a kódban csak a távolság felénél kevesebb hiba keletkezett, akkor az még mindig
közelebb lesz az eredeti kódszóhoz mint bármely másikhoz, így helyesen javítunk, azaz a $d$
távolsággal rendelkező kód pontosan $\lfloor\frac{d-1}{2}\rfloor$ hibajaító.

\begin{definition}{Kétdimenziós paritásbites kódolás}
    TODO
\end{definition}



\end{Section}

%\begin{Section}{
%  title/english=Crytography,
%  title/magyar=Kriptogárfia}
%
%  \input{number-theory/03-cryptography.tex}
%
%\end{Section}



\end{Part*}

%\begin{Part*}{title/magyar=Megoldások, title/english=Solutions}
%
%  \begin{section}{Számelmélet}
%
%    \subsection{} Sok-sok megoldás elképzelhet\H o, például:
%      \begin{sageexample}
%          sage: def divides0(a,b):
%          ....:     return (a/b).is_integer()
%          sage: def divides1(a,b):
%          ....:     return a % b == 0
%          sage: def divides2(a,b):
%          ....:     return (a//b)*b == a
%          sage: def divides3(a,b):
%          ....:     return (a/b).denom() == 1
%          sage: def divides4(a,b): #there is room to improve
%          ....:     if a == 0:
%          ....:         return True
%          ....:     b *= sign(b)
%          ....:     if b == 1:
%          ....:         return True
%          ....:     q = b
%          ....:     a *= sign(a)
%          ....:     while q <= a:
%          ....:         q <<= 1
%          ....:     while a > b:
%          ....:         q >>= 1
%          ....:         a -= q
%          ....:         a *= sign(a)
%          ....:     return a == 0 or a == b
%
%      \end{sageexample}
%  \end{section}
%
%\end{Part*}

\end{document}
